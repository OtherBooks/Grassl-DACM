\documentclass[10pt,letter]{article}

\usepackage{fullpage}
\usepackage{setspace}
\usepackage{parskip}
\usepackage{titlesec}
\usepackage{xcolor}
\usepackage{lineno}





\PassOptionsToPackage{hyphens}{url}
\usepackage[colorlinks = true,
            linkcolor = blue,
            urlcolor  = blue,
            citecolor = blue,
            anchorcolor = blue]{hyperref}
\usepackage{etoolbox}
\makeatletter
\patchcmd\@combinedblfloats{\box\@outputbox}{\unvbox\@outputbox}{}{%
  \errmessage{\noexpand\@combinedblfloats could not be patched}%
}%
\makeatother


\usepackage[round]{natbib}
\let\cite\citep


\renewenvironment{abstract}
  {{\bfseries\noindent{\large\abstractname}\par\nobreak}}
  {}

\renewenvironment{quote}
  {\begin{tabular}{|p{13cm}}}
  {\end{tabular}}

\titlespacing{\section}{0pt}{*3}{*1}
\titlespacing{\subsection}{0pt}{*2}{*0.5}
\titlespacing{\subsubsection}{0pt}{*1.5}{0pt}


\usepackage{authblk}
\makeatletter
\renewcommand\AB@authnote[1]{\rlap{\textsuperscript{\normalfont#1}}}
\renewcommand\Authsep{,~\,}
\renewcommand\Authands{,~\,and }
\makeatother


\usepackage{graphicx}
\usepackage[space]{grffile}
\usepackage{latexsym}
\usepackage{textcomp}
\usepackage{longtable}
\usepackage{tabulary}
\usepackage{booktabs,array,multirow}
\usepackage{amsfonts,amsmath,amssymb}
\providecommand\citet{\cite}
\providecommand\citep{\cite}
\providecommand\citealt{\cite}
% You can conditionalize code for latexml or normal latex using this.
\newif\iflatexml\latexmlfalse
\DeclareGraphicsExtensions{.pdf,.PDF,.png,.PNG,.jpg,.JPG,.jpeg,.JPEG}

\usepackage[utf8]{inputenc}
% \usepackage[ngerman,greek,english]{babel}







\begin{document}

\title{Discrete and Combinatorial Mathematics}



\author[1]{Oscar Levin}%
\affil[1]{Affiliation not available}%


\vspace{-1em}



  \date{\today}

\begingroup
\let\center\flushleft
\let\endcenter\endflushleft
\maketitle
\endgroup








MATH 528

Richard Grassl

University of Northern Colorado

School of Mathematical Sciences

\href{mailto:Richard.grassl@unco.edu}{\nolinkurl{Richard.grassl@unco.edu}}

Partially Funded by NSF Grant \#0832026

Revised August 2009

Table of Contents

A Brief History of Fibonacci
Numbers\ldots{}\ldots{}\ldots{}\ldots{}\ldots{}\ldots{}\ldots{}\ldots{}\ldots{}\ldots{}\ldots{}\ldots{}\ldots{}.\ldots{}.\ldots{}1

The Catalan
Numbers\ldots{}\ldots{}\ldots{}\ldots{}\ldots{}\ldots{}\ldots{}\ldots{}\ldots{}\ldots{}\ldots{}\ldots{}\ldots{}\ldots{}\ldots{}\ldots{}\ldots{}\ldots{}\ldots{}\ldots{}\ldots{}\ldots{}.7

Stirling
Numbers\ldots{}\ldots{}\ldots{}\ldots{}\ldots{}\ldots{}\ldots{}\ldots{}\ldots{}\ldots{}\ldots{}\ldots{}\ldots{}\ldots{}\ldots{}\ldots{}\ldots{}\ldots{}\ldots{}\ldots{}\ldots{}\ldots{}.\ldots{}.18

The Bell
Numbers\ldots{}\ldots{}\ldots{}\ldots{}\ldots{}\ldots{}\ldots{}\ldots{}\ldots{}\ldots{}\ldots{}\ldots{}\ldots{}\ldots{}\ldots{}\ldots{}\ldots{}\ldots{}\ldots{}\ldots{}...\ldots{}..
.. 24

Group
Projects\ldots{}\ldots{}\ldots{}\ldots{}\ldots{}\ldots{}\ldots{}\ldots{}\ldots{}\ldots{}\ldots{}\ldots{}\ldots{}\ldots{}\ldots{}\ldots{}\ldots{}\ldots{}\ldots{}\ldots{}\ldots{}\ldots{}.\ldots{}\ldots{}.33

Paths and Binomial
Coefficients\ldots{}\ldots{}\ldots{}\ldots{}\ldots{}\ldots{}\ldots{}\ldots{}\ldots{}\ldots{}\ldots{}.\ldots{}\ldots{}.\ldots{}..34

Triangular Numbers
\ldots{}\ldots{}\ldots{}\ldots{}\ldots{}\ldots{}\ldots{}\ldots{}\ldots{}\ldots{}\ldots{}\ldots{}\ldots{}\ldots{}\ldots{}..\ldots{}\ldots{}.\ldots{}....35

Fibonacci
Numbers\ldots{}\ldots{}\ldots{}\ldots{}\ldots{}\ldots{}\ldots{}\ldots{}\ldots{}\ldots{}\ldots{}\ldots{}\ldots{}\ldots{}\ldots{}\ldots{}\ldots{}\ldots{}\ldots{}\ldots{}.36

Generalized Pascal
Triangles\ldots{}\ldots{}\ldots{}\ldots{}\ldots{}\ldots{}\ldots{}\ldots{}\ldots{}\ldots{}\ldots{}\ldots{}\ldots{}\ldots{}..\ldots{}\ldots{}37

A Pascal-like Triangle of Eulerian
Numbers\ldots{}\ldots{}\ldots{}\ldots{}\ldots{}\ldots{}\ldots{}\ldots{}..\ldots{}\ldots{}38

Leibnitz Harmonic
Triangle\ldots{}\ldots{}\ldots{}\ldots{}\ldots{}\ldots{}\ldots{}\ldots{}\ldots{}\ldots{}\ldots{}\ldots{}\ldots{}\ldots{}\ldots{}..\ldots{}..39

The 12 Days of
Christmas\ldots{}\ldots{}\ldots{}\ldots{}\ldots{}\ldots{}\ldots{}\ldots{}\ldots{}\ldots{}\ldots{}\ldots{}\ldots{}\ldots{}\ldots{}\ldots{}..\ldots{}.40

Four (or Eight) Distribution
Problems\ldots{}\ldots{}\ldots{}\ldots{}\ldots{}\ldots{}\ldots{}\ldots{}\ldots{}\ldots{}\ldots{}\ldots{}\ldots{}\ldots{}.......41

Nineteen Combinatorial
Proofs\ldots{}\ldots{}\ldots{}\ldots{}\ldots{}\ldots{}\ldots{}\ldots{}\ldots{}\ldots{}\ldots{}\ldots{}\ldots{}\ldots{}\ldots{}\ldots{}\ldots{}
.....45

Linear
Partitions\ldots{}\ldots{}\ldots{}\ldots{}\ldots{}\ldots{}\ldots{}\ldots{}\ldots{}\ldots{}\ldots{}\ldots{}\ldots{}\ldots{}\ldots{}\ldots{}\ldots{}\ldots{}\ldots{}\ldots{}\ldots{}\ldots{}\ldots{}\ldots{}1-8

The Exponential Generating
Functions\ldots{}\ldots{}\ldots{}\ldots{}\ldots{}\ldots{}\ldots{}\ldots{}\ldots{}\ldots{}\ldots{}\ldots{}\ldots{}\ldots{}\ldots{}1-11

A BRIEF HISTORY OF FIBONACCI NUMBERS

Fibonacci numbers receive their name from Leonardo of Pisa (Leonardo
Pisano, c. 1175-1250), better known as Leonardo Fibonacci. Fibonacci is
a contraction of Filius Bonacci, son of Bonacci.

Leonardo was born about 1175 in the commercial center of Pisa. This was
a time of great interest and importance in the history of Western
Civilization. One finds the influence of the crusades stirring and
awakening the people of Europe by bringing them in contact with the more
advanced intellect of the East. During this time the Universities of
Naples, Padua, Paris, Oxford, and Cambridge were established, the Magna
Carta signed in England, and the long struggle between the Papacy and
the Empire was culminated. Commerce was flourishing in the Mediterranean
world and adventurous travelers such as Marco Polo were penetrating far
beyond the borders of the known world.

It is in this growing commercial activity that we find the young
Leonardo at Bugia on the Northern coast of Africa. Here the merchants of
Pisa and other commercial cities of Italy had large warehouses for the
storage of their goods. Actually very little is known about the life of
this great mathematician. No contemporary historian makes mention of
him, and one must look to his writings to find information about him.

A mathematician before his time, Leonardo of Pisa, alias Leonardo
Pisano, alias Leonardo Bigollo, alias Fibonacci, was a despair to his
teacher as a young boy and an enigma to his colleagues in his later
years. Convinced of the superiority of the Hindu-Arabic numeral system
over the Roman system, Leonardo wrote one of his greatest works,
\emph{Liber Abaci} (in English, ``Book of Calculating'') to introduce
this system to the Western world. The \emph{Liber Abaci} was written in
1202 but was not published until 1857 because ``it was too advanced for
Leonardo's contemporaries''. Along with the introduction and development
of many mathematical topics, the \emph{Liber Abaci} contained
interesting story problems that Leonardo liked to invent. His most
popular problem is the breeding pair of rabbits: ``How many pairs of
rabbits will there be after a year if it is assumed that every month
each pair produces one new pair, which begins to bear young two months
after its own birth?''

This problem generates the infinite sequence that bears his name because
his work is the earliest known recording. The Fibonacci sequence begins
with 1 and each number that follows is the sum of the previous two
numbers. The first ten Fibonnaci numbers are: 1, 1, 2, 3, 5, 8, 13, 21,
34, 55.

Leonardo, in 1228, gave a second edition of the \emph{Liber Abaci} which
he dedicated to Michel Scott, astrologer to the Emperor Frederic II and
author of many scientific works. Copies of this edition exist today.
Leonardo profusely illustrated and strongly advocated the Hindu-Arabic
system in this work. He gave an extensive discussion of the Rule of
False Position and the Rule of Three. Leonardo did not use a general
method in problem solving; each problem was solved independently of the
others. In the solution of a problem he not only considered the problem
as it might occur, but considered all of the variations of the question,
even those that were not reasonable.

Because of Leonardo's great reputation, the emperor Frederick II, when
in Pisa (1225), held a sort of mathematical tournament to test
Leonardo's skill. The competitors were informed beforehand of the
questions to be asked, some or all of which were composed by Johannes of
Palermo, who was of Frederick's staff. This is the first case in the
history of mathematics that one meets with an instance of these
challenges to solve particular problems which were so common in the
sixteenth and seventeenth centuries.

The first question propounded was to find a number of which the square
when decreased or increased by 5 would remain a square. The correct
answer given by Leonardo was 41/12. The next question was to find by the
methods used in the tenth book of Euclid a line whose length x should
satisfy the equation $x^3 + 2x^2 + 10x - 20 = 0$. Leonardo showed by
geometry that the problem was impossible, but he gave an approximation
of the root 1.3688081075\ldots{}, which is correct to nine places.
\emph{Liber Abaci} contains many additional problems of this type.

After the `Rabbit Problem', the matter lay for 400 years. In 1611,
Johann Kepler, of astronomy fame, arrived at the series 1, 1, 2, 3, 5,
8, 13, 21, \ldots{} . There is no indication that he had access to one
of Fibonacci's hand-written books (The \emph{Liber Abaci} was not
published until 1857).

\[
_{n + 2} = F_{n + 1} + F_{n}
\]

A hundred years must pass before the problem is again considered. In
1753, R. Simpson derived a formula, implied by Kepler

\[
_{n - 1}F_{n + 1} - F_{n}^{2} = \left( - 1 \right)^{n}
\]

A second hundred years pass by and the series again comes under study.
In 1843, J.P.M. Binet derives an analytical function for determining the
value of any Fibonacci number

2\textsuperscript{n}\(\ \sqrt{5}\) F\textsubscript{n} = (1
+\(\ \sqrt{5}\))\textsuperscript{n} - (1 -
\(\sqrt{5}\))\textsuperscript{n}

In 1846 E. Catalan derived the formula:

\[
^{n - 1}F_{n} = \frac{n}{1} + \frac{5n\left( n - 1 \right)\left( n - 2 \right)}{1 \bullet 2 \bullet 3} + \frac{5^{2}n\left( n - 1 \right)\left( n - 2 \right)\left( n - 3 \right)\left( n - 4 \right)}{1 \bullet 2 \bullet 3 \bullet 4 \bullet 5}
\]

By now, the series had received enough attention to deserve a name. It
was variously called the Braun series, the Schimper-Braun series, the
Lamé series and the Gerhardt series. A. Braun, applied the series to the
arrangement of the scales of pine cones. Schimper is completely unknown.
Gerhardt is probably a misspelling of Girard.

Edouard Lucas, who dominated the field of recursive series during the
period 1876-1891, first applied Fibonacci's name to the series and it
has been known as the Fibonacci series since then.

About this time, 1858, Sam Loyd claimed to have invented the
checkerboard paradox. It is first found in print in a German journal in
1868. Today it seems proper to call it the Carroll Paradox after Lewis
Carroll (Charles Dodgson, 1832-1893) who was quite fond of it.

PROBLEMS ON FIBONACCI NUMBERS
\begin{enumerate}
\def\labelenumi{\arabic{enumi}.}

\item
  Determine a formula for each:

\end{enumerate}
\begin{enumerate}
\def\labelenumi{(\alph{enumi})}

\item
  \(F_{0} + F_{1} + F_{2} + \ldots + F_{m}\) (b)
  \(F_{0} + F_{2} + F_{4} + \ldots + F_{m}\) where m is even.

\end{enumerate}
\begin{enumerate}
\def\labelenumi{\arabic{enumi}.}

\item
  Prove each of the following

\end{enumerate}
\begin{enumerate}
\def\labelenumi{(\alph{enumi})}

\item
  \(F_{n + 1}^{2} - \ F_{n}^{2} = \ F_{n - 1}F_{n + 2}\) (b)
  \(F_{k}^{2} = \ F_{k}(F_{k + 1} - \ F_{k - 1})\)

\end{enumerate}
\begin{enumerate}
\def\labelenumi{\arabic{enumi}.}

\item
  Determine a closed expression for F\textsubscript{0}F\textsubscript{3}
  + F\textsubscript{1}F\textsubscript{4} + \ldots{} +
  F\textsubscript{n-1}F\textsubscript{n+2} using 2(a).
\item
  Determine a formula for
  \(F_{1}^{2} + \ F_{2}^{2} + \ \ldots + \ F_{n}^{2}\) in two ways:

\end{enumerate}
\begin{enumerate}
\def\labelenumi{(\alph{enumi})}

\item
  Collect data and prove by mathematical induction.
\item
  Use 2(b) and telescoping sums.

\end{enumerate}
\begin{enumerate}
\def\labelenumi{\arabic{enumi}.}

\item
  Give a geometrical ``proof'' for the result in Problem 4.
\item
  Prove \(\begin{pmatrix}
  0 & 1
  1 & 1
  \end{pmatrix}^{n} = \begin{pmatrix}
  F_{n - 1} & F_{n}
  F_{n} & F_{n + 1}
  \end{pmatrix}\) by induction. For convenience let \(Q =\begin{pmatrix}
  0 & 1
  1 & 1
  \end{pmatrix}
\).
\item
  Prove that \(F_{n - 1}F_{n + 1} - F_{n}^{2} = {( - 1)}^{n}\ \) in two
  ways:

\end{enumerate}
\begin{enumerate}
\def\labelenumi{(\alph{enumi})}

\item
  Mathematical induction.
\item
  Use Problem 6 and determinants.

\end{enumerate}
\begin{enumerate}
\def\labelenumi{\arabic{enumi}.}

\item
  Is there a result analogous to that in Problem 7 for just the integers
  1, 2, 3, 4, \ldots{}?
\item
  Show that \(Q^{2n + 1} = Q^{n}Q^{n}\)\textsuperscript{+1} and that
  \(F_{2n + 1} = \ F_{n + 1}^{2} + \ F_{n}^{2}\).
\item
  Prove the Binet Formula: \(F_{n = \ }\frac{a^{n} - b^{n}}{a - b}\)
  where a, b are roots of x\textsuperscript{2} - x - 1 = 0.
\item
  Prove
  \(\left( \frac{n}{0} \right)F_{0} + \ \left( \frac{n}{1} \right)F_{1} + \ \ldots + \ \left( \frac{n}{n} \right)F_{n} = \ F_{2n}\)
  in two ways:

\end{enumerate}
\begin{enumerate}
\def\labelenumi{(\alph{enumi})}

\item
  Use the Binet Formula.
\item
  Use Q.

\end{enumerate}
\begin{enumerate}
\def\labelenumi{\arabic{enumi}.}

\item
  Prove that \(\sum_{n = 2}^{\infty}\frac{1}{F_{n - 1}F_{n + 1}} = 1.\)

\end{enumerate}
\begin{quote}
Hint: Show first that
\(\frac{1}{F_{n - 1}F_{n + 1}} = \ \frac{1}{F_{n - 1}F_{n}} - \ \frac{1}{F_{n}F_{n + 1}}\)

\end{quote}
\begin{enumerate}
\def\labelenumi{\arabic{enumi}.}

\item
  Prove that
  \(\sum_{n = 2}^{\infty}\frac{F_{n}}{F_{n - 1}F_{n + 1}} = 2.\)
\item
  Prove that
  \(\operatorname{}\frac{F_{n + 1}}{F_{n}} = \ \frac{1 + \ \sqrt{5}}{2}.\)
\item
  For which values of n is F\textsubscript{n} an integral multiple of 3?
\item
  Can you have four distinct positive Fibonacci numbers in arithmetic
  progression?
\item
  How many Fibonacci numbers are perfect squares?
\item
  Find a closed formula for
  \(\frac{1}{F_{1}} + \ \frac{1}{F_{2}} + \ \frac{1}{F_{4}} + \ \frac{1}{F_{8}} + \ \ldots.\)
\item
  Conjecture and prove a formula for
  \(\left( \frac{n}{0} \right) + \left( \frac{n - 1}{1} \right) + \left( \frac{n - 2}{2} \right) + \ldots.\)
\item
  In what sense is \(\frac{x}{1 - x - x^{2}}\) the generating function
  for the sequence of Fibonacci numbers?
\item
  Prove that \(F_{5n + 5} = 3F_{5n} + 5F_{5n + 1}\)
  Hint: Use \(Q^{5n + 5}\).
\item
  Let $\varphi$ be the positive root of x\textsuperscript{2} - x - 1 = 0. Prove
  that \(\ \varphi^{n} = F_{n}\varphi + F_{n - 1}\).
\item
  Can every positive integer be written as a sum of distinct Fibonacci
  numbers?

THE CATALAN NUMBERS

(Notes of a talk presented by Professor Richard Grassl, Muhlenberg
  College, March 15, 1991)

\end{enumerate}

\[
_{n + 1} = C_{1}C_{n} + C_{2}C_{n - 1} + \ldots + C_{n}C_{1}
\]
\begin{enumerate}
\def\labelenumi{\Roman{enumi}.}

\item
  \textbf{Parenthesizing:} In how many ways can you insert parentheses
  in the nonassociative product abcd? The five ways are displayed

(ab)(cd) ((ab)c)d a(b(cd)) (a(bc))d a((bc)d)

There are two ways for abc, namely a(bc) and (ab)c, and one for (ab).
  For a product of n symbols
  a\textsubscript{1}a\textsubscript{2}\ldots{}a\textsubscript{n}, break
  it at the k-th symbol:

\end{enumerate}

\[
\left( a_{1}a_{2}a_{3}\ldots a_{k} \right)(a_{k + 1}\ldots a_{n})
\]

Let P\textsubscript{n} denote the number of ways of inserting integers
into this product. There are P\textsubscript{k} ways of inserting
parentheses in \(\left( a_{1}\ldots a_{k} \right)\) and \(P_{n - k}\)
ways for the factor \((a_{k + 1}\ldots a_{n})\). Letting k range from 1
to \(n - 1\) we have that

\[
_{n} = \ \sum_{k = 1}^{n - 1}{P_{k}P_{n - k} = \ P_{1}P_{n - 1} + \ P_{2}P_{n - 2} + \ \ldots + \ P_{n - 1}P_{1}}
\]
\begin{enumerate}
\def\labelenumi{\Roman{enumi}.}

\item
  \textbf{The H-D sequences:} 2n people stand in line at a theatre.
  Admission is 50 cents, (denoted by H), and the box office starts with no
  change. n of the people have H and n have \$1(D). In how many ways can
  the 2n people line up so that all can be admitted? Here we enumerate
  the number of workable sequences of n H's and n D's such that at each
  point in the sequence the number of H's is not less than the number of
  D's.

The total number of sequences of n H's and n D's is
  \(\left( \frac{2n}{n} \right)\). We delete the number of
  \textbf{nonworkable} sequences. Each such sequence has a first snag as
  shown by the arrow in

H H D D D H H D D D H H

Reverse each letter up to and including the snag, obtaining

D D H H H H H D D D H H

a sequence of n+1 H's and \(n - 1\) D's. For \(n = 3\), H D D D H H is
  a nonworkable sequence with the snag indicated. Its mate, found by
  reversing the first three letters, is

D H H D H H, a sequence of 2D's and 4H's. \emph{Any} arrangement of
  2D's and 4H's will correspond to exactly one nonworkable sequence;
  simply scan through and see the \emph{first} time the H's dominate the
  D's and then reverse through that spot. There are
  \(\begin{pmatrix}
  6
  3
  \end{pmatrix}
 = 20\) arrangements of 3H's and 3D's. There are
  \(\begin{pmatrix}
  6
  2
  \end{pmatrix}
 = 15\) arrangements of 2D's and 4H's each of which
  corresponds to a nonworkable sequence. Hence there are
  \(\begin{pmatrix}
  6
  3
  \end{pmatrix}
 -\begin{pmatrix}
  6
  2
  \end{pmatrix}
 = 5\) workable sequences. In general,
  \(\left( \frac{2n}{n} \right) - \left( \frac{2n}{n - 1} \right) = \ \frac{1}{n + 1}\left( \frac{2n}{n} \right)\)
  gives the number of workable sequences. See FIGURE 1.

For n = 3, the matching of the 5 workable sequences with one of the 5
  ways of parenthesizing, given by the bijection ( H, letter  D is
  presented:

\end{enumerate}

H H H D D D   (((abcd

H H D H D D   ((a(bcd

H H D D H D   ((ab(cd

H D H H D D   (a((bcd

H D H D H D  (a(b(cd
\begin{enumerate}
\def\labelenumi{\Roman{enumi}.}

\item
  \textbf{Paths:} A point (a, b) in the plane is a lattice point if a
  and b are integers. How many paths of length 2n, consisting of
  horizontal and vertical segments of unit length, are there from (0, 0)
  to (n, n) such that the path never goes above the line y = x? One such
  path to (3, 3) is shown.

\end{enumerate}
\begin{longtable}[]{@{}lllll@{}}
\toprule
& & & (3,3) &\tabularnewline
\midrule
\endhead
& & & &\tabularnewline
& & & &\tabularnewline
& & & &\tabularnewline
\bottomrule

\end{longtable}
\begin{quote}
Using R for right and U for up, the sequence R U R R U U gives the path.
The bijection R H and U D links paths to H-D sequences.

\end{quote}
\begin{enumerate}
\def\labelenumi{\Roman{enumi}.}

\item
  \textbf{Triangulations of convex \emph{n}-gons:} Fix \emph{n}
  vertices; in how many ways can diagonals be inserted so as to
  decompose the \emph{n} -gon into triangles. For \emph{n} = 5 the five
  figures are

\end{enumerate}
\begin{quote}
There are two drawings for \emph{n} = 4

A bijection between triangulations and parenthesizing is illustrated
next:

\end{quote}
\begin{enumerate}
\def\labelenumi{\Roman{enumi}.}

\item
  \textbf{Tableau Insertion:} Insert the integers 1, 2, \ldots{},
  2\emph{n} into a 2 by \emph{n} rectangle of boxes such that the
  entries are monotonic in rows and columns. For \emph{n} = 3 there are
  five arrangements:

\end{enumerate}
\begin{longtable}[]{@{}ccccccccccccccccccc@{}}
\toprule
1 & 2 & 3 & & 1 & 2 & 4 & & 1 & 2 & 5 & & 1 & 3 & 4 & & 1 & 3 &
5\tabularnewline
\midrule
\endhead
4 & 5 & 6 & & 3 & 5 & 6 & & 3 & 4 & 6 & & 2 & 5 & 6 & & 2 & 4 &
6\tabularnewline
\bottomrule

\end{longtable}

A bijection to the path problem is: an entry in the top row   R. For
example, the middle arrangement corresponds to R R U U R U.
\begin{enumerate}
\def\labelenumi{\Roman{enumi}.}

\item
  \textbf{Trivalent rooted trees:} A tree is a connected graph that has
  no cycles. Trivalent means that each interior vertex has degree 3. The
  ``leaves'' or endpoints have degree 1. The number of trivalent rooted
  trees having n-vertices (not counting the root) is C\textsubscript{n}.
  The cases n = 1, 2, 3, 4 are drawn in FIGURE 2 along with bijections
  linking H-D sequences and parenthesizing. For the bijection, label
  interior vertices with a 1 and leaves with a 0. Start at the root,
  keep bearing to the right, and call off the sequence of 0's and 1's,
  not repeating them once called.

A link between a particular triangulation of a 5-gon and its
  associated tree is displayed in the following diagram:

Place a vertex in each triangle, connect the three vertices and run
  leaves out all edges, including the bottom which forms the root.
  Straighten the tree out! FIGURE 3 shows the five triangulations of a
  convex pentagon along with their trees.
\item
  \textbf{Rhyme schemes for \emph{n} line stanzas}

A 2-line poem can have just two rhyme schemes

\end{enumerate}
FIX
% \[
% begin{matrix}
% a \\
% a \\
% \end{matrix}\text{\ \ \ \ \ \ \ \ \ \ }\text{or}\text{\ \ \ \ \ \ \ \ \ \ }\begin{matrix}
% a \\
% b \\
% \end{matrix}
% \]

A 3-line poem can have five schemes:
FIX
% \[
% begin{matrix}
% a \\
% a \\
% a \\
% \end{matrix}\text{\ \ \ }\begin{matrix}
% a \\
% a \\
% b \\
% \end{matrix}\text{\ \ \ \ }\begin{matrix}
% a \\
% b \\
% a \\
% \end{matrix}\text{\ \ \ \ }\begin{matrix}
% a \\
% b \\
% b \\
% \end{matrix}\text{\ \ \ \ }\begin{matrix}
% a \\
% b \\
% c \\
% \end{matrix}
% \]

The last pattern indicates that no line rhymes with any other. The
Catalan numbers seem to be popping up again. For \emph{n} = 4 there
happen to be 15 (not 14) patterns:
\begin{longtable}[]{@{}ccccccccccccccc@{}}
\toprule
a & a & a & a & a & a & a & a & a & a & a & a & a & a & a\tabularnewline
\midrule
\endhead
a & a & a & b & b & a & b & b & a & b & b & b & b & b & b\tabularnewline
a & a & b & a & b & b & b & a & b & a & c & b & c & c & c\tabularnewline
a & b & a & a & b & b & a & b & c & c & a & c & b & c & d\tabularnewline
\bottomrule

\end{longtable}

One of these, abab, is special in that it is the only ``non-planar''
scheme. In this pattern, you cannot connect the a's with an arc and the
b's with an arc without the arcs crossing.

The planar rhyme schemes are enumerated by the Catalan numbers. These
numbers are a subsequence of the Bell numbers, which enumerate all rhyme
schemes.

Thus far, we have the following bijections:

Parentheses   H-D sequences

H-D   Paths

Triangulations   Parentheses

Insertions   Paths

Trees   Parentheses

See if you can connect the rhyme schemes to any of the above.

THE CATALAN NUMBERS
\begin{longtable}[]{@{}cccccccccccccccccccccc@{}}
\toprule
& & & & & & & & & & 1 & & & & & & & & & & & 1\tabularnewline
\midrule
\endhead
& & & & & & & & & 1 & & 1 & & & & & & & & & &\tabularnewline
& & & & & & & & 1 & & 2 & & 1 & & & & & & & & & 1\tabularnewline
& & & & & & & 1 & & 3 & & 3 & & 1 & & & & & & & &\tabularnewline
& & & & & & 1 & & 4 & & 6 & & 4 & & 1 & & & & & & & 2\tabularnewline
& & & & & 1 & & 5 & & 10 & & 10 & & 5 & & 1 & & & & & &\tabularnewline
& & & & 1 & & 6 & & 15 & & 20 & & 15 & & 6 & & 1 & & & & &
5\tabularnewline
& & & 1 & & 7 & & 21 & & 35 & & 35 & & 21 & & 7 & & 1 & & &
&\tabularnewline
& & 1 & & 8 & & 28 & & 56 & & 70 & & 56 & & 28 & & 8 & & 1 & & &
14\tabularnewline
& 1 & & 9 & & 36 & & 84 & & 126 & & 126 & & 84 & & 36 & & 9 & & 1 &
&\tabularnewline
1 & & 10 & & 45 & & 120 & & 210 & & 252 & & 210 & & 120 & & 45 & & 10 &
& 1 & 42\tabularnewline
\bottomrule

\end{longtable}

FIGURE 1

TRIVALENT, PLANTED TREES WITH n LEAVES

n=1 n=2 n=3

n=4

1100100 1010100

( ( ab ( cd ( a ( b ( cd

HHDDHD HDHDHD

FIGURE 2
\begin{longtable}[]{@{}ccc@{}}
\toprule\begin{minipage}[b]{0.32\columnwidth}\centering\strut
\strut
\end{minipage} & \begin{minipage}[b]{0.32\columnwidth}\centering\strut
\strut
\end{minipage} & \begin{minipage}[b]{0.32\columnwidth}\centering\strut
((ab(cd))

1100100\strut
\end{minipage}\tabularnewline
\midrule
\endhead\begin{minipage}[t]{0.32\columnwidth}\centering\strut
\strut
\end{minipage} & \begin{minipage}[t]{0.32\columnwidth}\centering\strut
\strut
\end{minipage} & \begin{minipage}[t]{0.32\columnwidth}\centering\strut
(((ab)c)d)

1110000\strut
\end{minipage}\tabularnewline\begin{minipage}[t]{0.32\columnwidth}\centering\strut
\strut
\end{minipage} & \begin{minipage}[t]{0.32\columnwidth}\centering\strut
\strut
\end{minipage} & \begin{minipage}[t]{0.32\columnwidth}\centering\strut
(a(b(cd)))

1010100\strut
\end{minipage}\tabularnewline\begin{minipage}[t]{0.32\columnwidth}\centering\strut
\strut
\end{minipage} & \begin{minipage}[t]{0.32\columnwidth}\centering\strut
\strut
\end{minipage} & \begin{minipage}[t]{0.32\columnwidth}\centering\strut
(a(bc)d))

1011000\strut
\end{minipage}\tabularnewline\begin{minipage}[t]{0.32\columnwidth}\centering\strut
\strut
\end{minipage} & \begin{minipage}[t]{0.32\columnwidth}\centering\strut
\strut
\end{minipage} & \begin{minipage}[t]{0.32\columnwidth}\centering\strut
((a(bc))d)

1101000\strut
\end{minipage}\tabularnewline
\bottomrule

\end{longtable}

FIGURE 3

PROBLEMS ON CATALAN NUMBERS
\begin{enumerate}
\def\labelenumi{\arabic{enumi}.}

\item
  Use the recursion for P\textsubscript{n} and compute
  P\textsubscript{6}, P\textsubscript{7}.
\item
  Use the closed formula for P\textsubscript{n} and compute
  P\textsubscript{6}, P\textsubscript{7}.
\item
  Use G = P\textsubscript{1}x + P\textsubscript{2}x\textsuperscript{2} +
  \ldots{} and the recursion for P\textsubscript{n} to determine a
  closed formula for P\textsubscript{n}.
\item
  List the workable sequences for n = 4.
\item
  List the P\textsubscript{6} ways of parenthesizing abcdef.
\item
  Display a bijection between the 14 sequences in Problem 4 and the 14
  products in Problem 5.
\item
  Find the nonworkable sequence associated with each:

\end{enumerate}
\begin{enumerate}
\def\labelenumi{(\alph{enumi})}

\item
  H H H D D H (c) D D D H H H H H
\item
  D H H H H D (d) D D H H D H H H

\end{enumerate}
\begin{enumerate}
\def\labelenumi{\arabic{enumi}.}

\item
  Draw the T\textsubscript{6} triangulations of a convex hexagon.
\item
  Display a bijection between the 5 ways of parenthesizing abcd and the
  5 triangulations of a convex pentagon.
\item
  Let \(C_{n} = \ \frac{1}{n + 1}\left( \frac{2n}{n} \right).\) Verifiy
  the formula:

\end{enumerate}

\[
_{n} = \ \left( \frac{n}{1} \right)C_{n - 1} - \ \left( \frac{n - 1}{2} \right)C_{n - 2} + \ \left( \frac{n - 2}{3} \right)C_{n - 3} - \ \ldots
\]
\begin{enumerate}
\def\labelenumi{\arabic{enumi}.}

\item
  Verify that H\textsubscript{3} = 5 by drawing the required paths.
\item
  Assign the appropriate binary sequences to each of the 5 Trees for n=3
  in FIGURE 2.

\end{enumerate}

FOR EACH OF THE FOLLOWING, INVESTIGATE BIJECTIONS THAT RELATE ONE TO
ANOTHER.
\begin{enumerate}
\def\labelenumi{\arabic{enumi}.}

\item
  A and B each receive n votes. Let V\textsubscript{n} denote the number
  of ways that the 2n votes can be tallied so that A never trails B. Let
  V\textsubscript{0} = 1.
\item
  Place 2n points on the circumference of a circle and draw n
  nonintersecting chords in D\textsubscript{n} ways; D\textsubscript{0}
  = 1.
\item
  In how many ways can 1, 2, 3, \ldots{}, 2n be inserted into a 2xn
  rectangle such that the entries are increasing in rows and columns.
  Let the answer be Y\textsubscript{n}; Y\textsubscript{1} = 1,
  Y\textsubscript{2} = 2.
\item
  Place points on a line segment and join them in pairs by
  nonintersecting arcs above the segment. In how many ways can this be
  done? Call the answer S\textsubscript{n}; S\textsubscript{1}=1,
  S\textsubscript{2}=2.
\item
  A rook on an \(n + 1\) by \(n + 1\) chessboard must move from the
  lower left corner to the upper right corner never going above the
  diagonal. How many paths are possible? Let \(K_{n}\) be the answer.
\item
  On an \(n + 1\) by \(n + 1\) chessboard a king starts on the first row
  and moves one square forward or back along a fixed column and ends on
  the starting square after moves. In how many ways, \(G_{n}\), can this
  be done?
\item
  Count the number of planar rhyme schemes for a stanza consisting of n
  lines. Joanne Growney showed in her doctoral thesis that the Bell
  numbers, which count all rhyme schemes, have the Catalan numbers as a
  subsequence and that these enumerate precisely the planar rhyme
  schemes. For example, of the rhyme schemes, are planar. The nonplanar
  one is given by abab.

\end{enumerate}

WORKSHEET ON STIRLING NUMBERS OF THE 2\textsuperscript{ND}
KIND-\(S(n,\ k)\)

\emph{DEFINITION:} \(S\left( n,\ k \right)\) is the number of ways of
partitioning an n-set into k nonempty subsets.

\emph{TASK 1:} Compute \(S\left( 3,\ 1 \right),\ S(4,\ 1)\) and
\(S(n,\ 1)\).

\emph{TASK 2:} Compute \(S\left( 2,\ 2 \right),\ S(3,\ 2)\) and
\(S(4,\ 2)\).

\emph{TASK 3:} Compute\(\text{\ S}\left( 3,\ 3 \right),\ S(4,\ 3)\).

\emph{TASK 4:} Find formulas and give \emph{proofs} for:

\(S\left( n,\ n \right) =\)

\(S\left( n,\ 2 \right) =\)

\[
\left( n,\ n - 1 \right) =
\]

\[
\left( n,\ n - 2 \right) =
\]

\emph{TASK 5:} Determine a recursion for
\(\text{\ S}\left( n,\ k \right)\) by examining the following:

In computing \(S\left( 4,\ 3 \right)\) look back at the two cases:
\begin{enumerate}
\def\labelenumi{(\alph{enumi})}

\item
  ``Social''- The element 4 is part of another set.
\item
  ``Antisocial''-The element 4 stands alone.
\item
  Deduce a recursion \(S\left( n,\ k \right) =\)

\end{enumerate}

\emph{TASK 6:} Make the first 5 rows of the Stirling Triangle.

STIRLING NUMBERS OF THE 2\textsuperscript{nd} KIND

A distribution of 4 different (often we say distinguishable) objects
\(1,2,3,4\) into 3 similar (often we say indistinguishable) boxes, each
of which becomes non-empty is modeled by partitioning \(\{ 1,2,3,4\}\)
into 3 nonempty subsets. The six ways of doing this are listed next:

\{4\} \{1\} \{2,3\} \{1,4\} \{2\} \{3\}

\{4\} \{2\} \{1,3\} \{1\} \{2,4\} \{3\}

\{4\} \{3\} \{1,2\} \{1\} \{2\} \{3,4\}

The Stirling numbers of the second kind, denoted by \(S(n,k)\), give the
number of ways of partitioning an n-set into k non-empty subsets. Here
\(k \in \{ 1,2,\ldots\}\) and\(\ n \geq k\). The above example shows
\(S\left( 4,3 \right) = 6\). Also notice that there are two cases: 4 is
alone, or 4 is in some other batch; we can conclude that
\(S\left( 4,3 \right) = 3S\left( 3,3 \right) + S(3,2)\) since if 4 is in
a singleton set just partition the remaining 3 into 2 sets and otherwise
4 can join any of the 3 batches of 3 elements.

THEOREM 1:
\(S\left( n,k \right) = kS\left( n - 1,k \right) + S(n - 1,k - 1)\).

Proof. We partition \(\{ 1,2,\ldots,n\}\) into k non-empty subsets in
\(S(n,k)\) ways. If n is in a singleton set, there are
\(S(n - 1,k - 1)\) ways to partition the remaining \(n - 1\) elements
into \(k - 1\) subsets. Otherwise, the element n can join any of the k
batches of \(n - 1\) elements in \(kS(n - 1,k)\) ways.

Polya, in Notes on Introductory Combinatorics, refers to these as the
``antisocial'' and ``social'' cases.

The reader should check that \(S\left( 3,3 \right) = 1\) and that
\(S\left( 3,2 \right) = 3\) so that
\(S\left( 4,3 \right) = 3 \bullet 1 + 3 = 6\).

\emph{Border Formulas:} The reader should verify that
\(S\left( n,1 \right) = 1\) and \(S\left( n,n \right) = 1\) for
\(n \in Z^{+}\). From

these border formulas and the recursion in Theorem 1 the following
Stirling's Second Triangle can be made.

1

1 1

1 3 1

1 7 6 1

1 15 25 10 1

THEOREM 2.
\begin{enumerate}
\def\labelenumi{(\alph{enumi})}

\item
  \(S\left( n,n - 1 \right) =\begin{pmatrix}
  n
  2
  \end{pmatrix}
\)
\item
  \(S\left( n,2 \right) = 2^{n - 1} - 1\)
\item
  \(S\left( n,n - 2 \right) =\begin{pmatrix}
  n
  3
  \end{pmatrix}
 + 3\begin{pmatrix}
  n
  4
  \end{pmatrix}
\)
\item
  \(S\left( n,n - 3 \right) =\begin{pmatrix}
  n
  4
  \end{pmatrix}
 + 10\begin{pmatrix}
  n
  5
  \end{pmatrix}
 + 15\begin{pmatrix}
  n
  6
  \end{pmatrix}
\)

\end{enumerate}

Proof. You are asked to prove these in the problems.

The entries in the Pascal Triangle, \(\begin{pmatrix}
n
k
\end{pmatrix}
\), satisfy the nice recursion \(\begin{pmatrix}
n
k
\end{pmatrix}
 =
\begin{pmatrix}
n - 1
k
\end{pmatrix}
 +\begin{pmatrix}
n - 1
k - 1
\end{pmatrix}
\) and also have a closed formula: \(\begin{pmatrix}
n
k
\end{pmatrix}
 = n!/k!\left( n - k \right)!\). Theorem 1 gives us a
recursion for the Stirling numbers \(S\left( n,k \right)\ \)and,
unfortunately, the best we can do for a ``closed'' formula is contained
in the following.

\[
\left( n,k \right) = \frac{1}{k!}\left\lbrack k^{n} - \begin{pmatrix}
k \\
1 \\
\end{pmatrix}\left( k - 1 \right)^{n} + \begin{pmatrix}
k \\
2 \\
\end{pmatrix}\left( k - 2 \right)^{n} - \begin{pmatrix}
k \\
3 \\
\end{pmatrix}\left( k - 3 \right)^{n} + \ldots + \left( - 1 \right)^{k - 1}\begin{pmatrix}
k \\
k - 1 \\
\end{pmatrix}1^{n} \right\rbrack
\]

Proof (by Polya). Suppose you wished to paint n houses and you have k
different colors available. The first house can be painted in k
different ways, the second in k different ways, etc. So there are
\(k^{n}\) ways. How many ways actually use all k colors? Let
\(\alpha_{i}\) be the property that no house is painted with the
\(i^{th}\)color, and let \(N(\alpha_{i})\) denote the number of ways of
painting the n houses without using the \(i^{th}\)color. Analogously for
\(\text{\ N}\left( \alpha_{i},\alpha_{j} \right),\ \ N\left( \alpha_{i},\alpha_{j},\ \alpha_{k} \right),\ \)
etc. Since \(N\left( \alpha_{i} \right) = \left( k - 1 \right)^{n}\),
\(N\left( a_{i},\ \alpha_{j} \right) = \left( k - 2 \right)^{n},\ \ldots\ \)
the number of ways of painting n houses using all k colors is, using
PIE, \(k^{n} -\begin{pmatrix}
k
1
\end{pmatrix}
\left( k - 1 \right)^{n} +
\begin{pmatrix}
k
2
\end{pmatrix}
\left( k - 2 \right)^{n} -
\begin{pmatrix}
k
3
\end{pmatrix}
\left( k - 3 \right)^{n} + \ \ldots + \left( - 1 \right)^{k}
\begin{pmatrix}
k
k
\end{pmatrix}
0^{n}\). Alternately, we could first partition the n houses
into k (non-empty) sets, and then paint each set. This can be
accomplished in \(k!S\left( n,k \right)\ \)ways. Now equate these two
expressions.

COROLLARY.
\(S\left( n,3 \right) = \frac{1}{2}\left\lbrack 3^{n - 1} - 2^{n} + 1 \right\rbrack\)

THEOREM 4.
\(x^{n} = S\left( n,1 \right)x + S\left( n,2 \right)x\left( x - 1 \right) + S\left( n,3 \right)x\left( x - 1 \right)\left( x - 2 \right) + \ \ldots\ \)

\[
ldots + S\left( n,n \right)x\left( x - 1 \right)\ldots(x - n + 1)
\]

Proof (by Polya). We can paint n houses, with x colors available, in
\(x^{n}\) ways. Using exactly one color, there are
\(S\left( n,1 \right)x\) ways. Using exactly two colors there are
\(S\left( n,2 \right)x(x - 1)\)ways. Continue in this manner.

It is in this form that James Stirling originally developed these
numbers. \(S(n,\ k)\)is used to convert from powers to binomial
coefficients as shown in the following:

THEOREM 5. \(x^{n} = S\left( n,1 \right)
\begin{pmatrix}
x
1
\end{pmatrix}
1! + S\left( n,2 \right)
\begin{pmatrix}
x
2
\end{pmatrix}
2! + \ \ldots + S\left( n,n \right)
\begin{pmatrix}
n
n
\end{pmatrix}
n!\ \)

PROBLEMS
\begin{enumerate}
\def\labelenumi{\arabic{enumi}.}

\item
  Compute by listing the subsets \(S(4,2)\) and \(S(5,2)\).
\item
  Make the 6\textsuperscript{th} row of the Stirling Triangle.
\item
  Prove the four parts of THEOREM 2.
\item
  List the sequence of elements that are row sums of the Stirling
  Triangle.
\item
  How many subsets \{a, b, c\} are there of \{2, 3, 4, \ldots{}\} such
  that
  \(abc = 2 \bullet 3 \bullet 5 \bullet 7 \bullet 11 \bullet 13 \bullet 17\)?
\item
  Let \(S = \{ 2,3,4,\ldots\}\). How many ordered triples (a, b, c) are
  there in S x S x S such that

\(abc = 2 \bullet 3 \bullet 5 \bullet 7 \bullet 11 \bullet 13 \bullet 17\)?
\item
  Compute each using THEOREM 3.
\begin{enumerate}
  \def\labelenumii{\alph{enumii}.}

  \item
    \(S\left( 5,3 \right)\)
  \item
    \(S(6,3)\)
  \item
    \(S(n,2)\)
  \item
    \(S(n,3)\)

\end{enumerate}
\item
  Write out the proof of THEOREM 3 for k=3.
\item
  Express x\textsuperscript{3} as suggested in THEOREM 4. Also, multiply
  your answer out to verify.
\item
  Use your result in Problem 9 to express x\textsuperscript{3} as a
  linear combination of the binomial coefficients \(\begin{pmatrix}
  x
  1
  \end{pmatrix}
,\
\begin{pmatrix}
  x
  2
  \end{pmatrix}
,\
\begin{pmatrix}
  x
  3
  \end{pmatrix}
\).
\item
  Determine the following:
\begin{enumerate}
  \def\labelenumii{\alph{enumii}.}

  \item
    \(a,\ b,\ c\) so that \(n^{4} = 24
\begin{pmatrix}
    n
    4
    \end{pmatrix}
 + 6a\
\begin{pmatrix}
    n
    3
    \end{pmatrix}
 + 2b\
\begin{pmatrix}
    n
    2
    \end{pmatrix}
 + c
\begin{pmatrix}
    n
    1
    \end{pmatrix}
\).
  \item
    \(a,\ b,\ c,\ d\) so that \(n^{5} = 5!
\begin{pmatrix}
    n
    5
    \end{pmatrix}
 + a
\begin{pmatrix}
    n
    4
    \end{pmatrix}
 + b
\begin{pmatrix}
    n
    3
    \end{pmatrix}
 + c
\begin{pmatrix}
    n
    2
    \end{pmatrix}
 + d
\begin{pmatrix}
    n
    1
    \end{pmatrix}
\).

\end{enumerate}
\item
  Express \(1^{4} + 2^{4} + 3^{4} + \ \ldots + n^{4}\) as a polynomial
  in n.
\item
  The numbers \(S(n,k)\) and \(k!S(n,k)\) are solutions to two different
  distribution problems. Describe each.
\item
  Prove THEOREM 5.
\item
  Why is \(4^{n} - 4 \bullet 3^{n} + 6 \bullet 2^{n} - 4\) always
  divisible by 24?

\end{enumerate}

BELL NUMBERS

The Bell number, \(B_{n}\), denotes the number of ways that a set of n
objects S can be partitioned into nonempty subsets. The definition of
``partition'' implies two properties: The subsets are disjoint and their
union is S.

Essential properties and facts about \(B_{n}\) are summarized next.

BELL NUMBERS

1, 1, 2, 5, 15, 52, 203, 877, 4040, 21147, 115975, \ldots{}

FORMULA:

\(B_{n} = S\left( n,1 \right) + \ S\left( n,2 \right) + S\left( n,3 \right) + \ \ldots + S(n,n)\)

\(B_{n} = \frac{1}{e}\sum_{k = 0}^{\infty}\frac{k^{n}}{k!}\ (Dobinski^{'}s\ Formula)\)

\(B_{n} = L\left( x^{n} \right)\ (Gian - Carlo\ Rota)\)

RECURSION: \(B_{n} =
\begin{pmatrix}
n - 1
0
\end{pmatrix}
B_{0} +
\begin{pmatrix}
n - 1
1
\end{pmatrix}
B_{1} + \ \ldots +
\begin{pmatrix}
n - 1
n - 1
\end{pmatrix}
B_{n - 1},\ B_{0} = 1\)

E.G.F.:
\(e^{e^{x} - 1} = \sum_{k = 0}^{\infty}{B_{n}\ \frac{x^{n}}{n!}}\)

BELL TRIANGLE: 1

1 2

2 3 5

5 7 10 15

15 20 27 37 52

\ldots{}

Two observations: The first few Bell's look like the Catalans; the
sequence of Bells increases in size rapidly. The Bell numbers can be
generated by constructing what is called the Bell Triangle. To construct
this triangle, begin with a 1 at the top and a 1 below it. Add these two
numbers together and put the sum 2, to the right of the 1 in the second
column. This 2 is also the first entry of the third row. The second
entry in the third row is found by adding the 2 to the number 1 above
it. This sum is 3 and goes to the right of the 2. The 3 is now below a
2. Adding these two numbers produces the last number, 5, in the third
row. Since 5 has no number above it, the third row is complete. 5 now
becomes the first number in the fourth row and the process continues.

Construction of the triangle follows two basic rules:
\begin{enumerate}
\def\labelenumi{\arabic{enumi}.}

\item
  The last number of each row is the first number of the next row.
\item
  All other numbers are found by adding the number to the left of the
  missing number to the number directly above this same number.

\end{enumerate}

When the triangle is extended, as above, the Bell Numbers are found down
the first column as well as along the outside diagonal.

1

1 2

2 3 5

5 7 10 15

15 20 27 37 52

52 67 87 114 151 203

203 255 322 409 523 674 877

As with Pascal's Triangle, the Bell triangle has several interesting
properties. If the sum of a row is added to the Bell number at the end
of that row, the next Bell number is obtained. For example, the sum of
the fourth row plus the Bell number at the end of the row:

15 + 20 + 27 + 37 + 52 + 52 is 203, the next Bell number. Also, the
numbers of the second diagonal 1, 3, 10, 37, 151, 674, \ldots{} are the
sums of the horizontal rows. Rotating the triangle slightly creates a
difference triangle analogous to Pascal's Triangle. The entries that are
formed recursively by adding in Pascal's Triangle are now differences of
the two numbers above them in the Bell triangle.

1 2 5 15 52 203 877\ldots{}

1 3 10 37 151 674\ldots{}

2 7 27 114 523\ldots{}

5 20 87 409\ldots{}

15 67 322\ldots{}

52 255\ldots{}

203\ldots{}

Finally, rewriting the Bell triangle recursively indicates a nice
connection of the Bell Numbers to Pascal's triangle.

\(B_{0} = B_{1}\)

\(B_{1}\) \(B_{0} + B_{1} = B_{2}\)

\(B_{2}\) \(B_{1} + B_{2}\) \(B_{0} + 2B_{1} + B_{2} = B_{3}\)

\(B_{3}\) \(B_{2} + B_{3}\) \(B_{1} + B_{2} + B_{2} + B_{3}\)
\(B_{0} + 3B_{1} + 3B_{2} + B_{3} = B_{4}\)

This pattern suggests the \({(n + 1)}^{\text{th}}\) Bell Number can be
represented recursively by

\(B_{n + 1} =
\begin{pmatrix}
n
0
\end{pmatrix}
B_{0} +
\begin{pmatrix}
n
1
\end{pmatrix}
B_{1} +
\begin{pmatrix}
n
2
\end{pmatrix}
B_{2} + \ \ldots +
\begin{pmatrix}
n
n
\end{pmatrix}
B_{n}\). The coefficients of this equation are the entries
of the nth row of Pascal's triangle. You are asked to prove this
recursion in the exercises.

The Bell numbers are related to the Stirling numbers and can be defined
as the sum of the Stirling numbers of the second kind. That is,
\(B_{n} = S\left( n,1 \right) + S\left( n,2 \right) + \ \ldots + S(n,n)\),
where

S(n, k) represents the number of ways of grouping n elements into k
subsets. So

\(B_{3} = S\left( 3,1 \right) + S\left( 3,2 \right) + S\left( 3,3 \right) = 1 + 3 + 1 = 5.\)
Since the Bell numbers count the partitions of a set of elements, they
are used in prime-number theory to enumerate the number of ways to
factor a number with distinct prime factors. For example, 42 has three
distinct prime factors: 2, 3, and 7. Since \(B_{3} = 5\), we know there
are five ways of factoring 42. These are

2 x 3 x 7, 2 x 21, 3 x 14, 6 x 7, and 42. So the number 210, which has 4
distinct factors, 2, 3, 5, and 7, can be factored in \(B_{4} = 15\)
ways.

The Bell numbers can be used to model many real-life situations. For
example, the number of different ways two people can sleep in unlabeled
twin beds is \(B_{2} = 2\) ways: They can sleep in the same bed or in
separate beds. The number of ways of serving a dinner consisting of
three items, such as a salad, bread, and fish is \(B_{3} = 5\) ways:
each could be served on a separate plate, salad and bread could be on
one plate and fish on another, salad and fish on one plate and bread on
another, or all three items could be on the same plate. This example
serves as a model for the five ways three people can occupy three
unlabeled beds, the five ways three prisoners can be handcuffed
together, the five ways three nations can be form alliances, or any
situation of partitioning three distinct elements into non-empty
subsets.

One interesting application of Bell numbers is in counting the number of
rhyme schemes possible for a stanza in poetry. There are \(B_{2} = 2\)
possibilities for a two-line stanza: the lines can either rhyme or not
rhyme. The possible rhyme schemes of a three-line stanza can be
described as aaa, aab, aba, abb, and abc: thus there are 5 or \(B_{3}\)
possible rhyme schemes. The Japanese used diagrams depicting possible
rhyme schemes of a five-line stanza \(B_{5} = 52\ \)as early as 1000
A.D. in the \emph{Tale of the Genji} by Lady Shikibu Murasaki.

PROBLEMS FOR BELL NUMBERS

The Bell number (named after Eric Temple Bell, a Scottish-born American
mathematician, who died in 1960) \(B_{n}\) is the number of ways of
partitioning an n-set into subsets. For example, \(B_{3} = 5\); the 5
ways of partitioning the elements of \{1, 2, 3\} are

\{1\} \{2\} \{3\} \{1, 2\} \{3\} \{1, 3\} \{2\} \{2, 3\} \{1\} \{1, 2,
3\}
\begin{enumerate}
\def\labelenumi{\arabic{enumi}.}

\item
  Compute \(B_{1},B_{2},\) and \(B_{4}\) by listing all the relevant
  partitions.
\item
  Explain why
  \(B_{n} = S\left( n,1 \right) + S\left( n,2 \right) + \ \ldots + S(n,n)\).
\item
  Compute \(B_{5},B_{6},B_{7},\ B_{8}\) using a table of Stirling
  numbers \(S(n,k)\) and the formula in Problem 2.
\item
  Prove that \(B_{n} =
\begin{pmatrix}
  n - 1
  0
  \end{pmatrix}
B_{0} +
\begin{pmatrix}
  n - 1
  1
  \end{pmatrix}
B_{1} +
\begin{pmatrix}
  n - 1
  2
  \end{pmatrix}
B_{2} + \ \ldots + \
\begin{pmatrix}
  n - 1
  n - 1
  \end{pmatrix}
B_{n - 1}\).

\end{enumerate}
\begin{quote}
{[}Here we let \(B_{0} = 1\){]}. HINT: First try n=5 and partition
\(\{ 1,2,3,4,\ 5\}\) by treating 5 as ``special.'' The element 5 can be
a singleton, in a doubleton, or \ldots{} .

\end{quote}
\begin{enumerate}
\def\labelenumi{\arabic{enumi}.}

\item
  Prove that

\end{enumerate}
\begin{quote}
\(S_{n}\left( n + 1,\ r \right) =
\begin{pmatrix}
n
0
\end{pmatrix}
S\left( 0,\ r - 1 \right) +
\begin{pmatrix}
n
1
\end{pmatrix}
S\left( 1,\ r - 1 \right) +
\begin{pmatrix}
n
2
\end{pmatrix}
S\left( 2,\ r - 1 \right) + \ \ldots\)

\(+
\begin{pmatrix}
n
n
\end{pmatrix}
S(n,\ r - 1)\)

\end{quote}
\begin{enumerate}
\def\labelenumi{\arabic{enumi}.}

\item
  Using the .results in Problems 2 and 5 prove the recursion in Problem
  4.
\item
  The set \(\left\{ u_{0},\ u_{1},\ u_{2},\ldots \right\}\) where
  \(u_{0} = 1,\ \text{\ u}_{1} = x,\ \ u_{2} = x\left( x - 1 \right),\ u_{3} = x\left( x - 1 \right)\left( x - 2 \right),\ \ldots\)
  is a basis for the vector space V of all polynomials with real
  coefficients. Let

\end{enumerate}
\begin{quote}
\(P\left( x \right) = c_{0}u_{0} + c_{1}u_{1} + c_{2}u_{2} + \ \ldots\ \)
be any element of V, and define the functional L as follows:
\(L\left\lbrack P\left( x \right) \right\rbrack = c_{0} + c_{1} + \ \ldots\)
.

\end{quote}
\begin{enumerate}
\def\labelenumi{\alph{enumi}.}

\item
  Prove that
  \(L\left\lbrack P\left( x \right) + Q\left( x \right) \right\rbrack = L\left\lbrack P\left( x \right) \right\rbrack + L\left\lbrack Q\left( x \right) \right\rbrack.\)
\item
  Prove that
  \(L\left\lbrack \text{rP}\left( x \right) \right\rbrack = rL\lbrack P\left( x \right)\rbrack\)
  for \(r \in\) Reals.
\item
  Prove that \(B_{n} = L\lbrack x^{n}\rbrack\).
\item
  Prove that
  \(L\left\lbrack x^{n + 1} \right\rbrack = L\lbrack\left( x + 1 \right)^{n}\rbrack\).
\item
  Derive the recursion in Problem 4 using (d).

\end{enumerate}

Tales of Statisticians

ERIC TEMPLE BELL

7 Feb 1883-20 Dec 1960

Bell was no statistician; rather, a number theorist. He was and remains
one of the great mathematical statesmen. His book \emph{Men of
Mathematics} has inspired scores to take up a mathematical career, and
revealed to uncounted thousands what a mathematical life is all about in
the first place.

His own early life was a mystery until parts of it were unraveled by
\emph{Constance Reid} . He was born in Aberdeen in 1883, spent his early
years in London, and came to America at the age of nineteen in 1903. He
finished college at Stanford in 1904, and went on to the University of
Washington (MS 1907) and Columbia (PhD 1912). He taught at the
University of Washington from that year until1926. His contribution to
statistics in this period was to encourage Harold Hotelling to switch
from journalism to math (Hotelling named his son after Bell). In his own
researches at this time, Bell produced significant results in number
theory, including Diophantine analysis: equations whose solutions are
limited to whole numbers. The Bell numbers (1, 1, 2, 5, 15, 52, and so
on), a series of whole numbers arising in the theory of partitions, were
first investigated by him, and are named for him. His 1921 memoir
Arithmetical Paraphrases won him the prestigious Bocher Prize (jointly
with Solomon Lefschetz) in 1924.

In that year he began a second life as a science fiction writer, under
the pen name John Taine (his only son, born in 1917, had been named
Taine Bell). In all, he produced twelve novels and several stories
between 1924 and 1954. In 1926, having earlier declined offers from
Chicago and Columbia, he joined the faculty of the California Institute
of Technology, where he became a distinctive and even eccentric figure,
and had much to do with building up the quality of the mathematics
faculty. His next mathematical publication was the comprehensive survey
Algebraic Arithmetic (1927). His enduring masterpiece of popularization,
\emph{Men of Mathematics} , came somewhat later, in 1937. The more
technically advanced \emph{Development of Mathematics} (1940) had a
comparable appeal and influence for professional and preprofessional
readers. His gift of explanation and his fascination with the integers
are both evident in the \emph{Magic of Numbers} (1946), a book on
Pythagoras and the number mysticism of the Pythagorean school.

Bell retired from CalTech in 1951, and died at the age of 77 in 1960.
With the assistance of D.H. Lehmer, who had provided the otherwise
missing final chapter, he had just completed a book entitled The Last
Problem. It was about Fermat, to whom, and to whose tantalizing unproven
Last Theorem, a chapter of \emph{Men of Mathematics} had already been
devoted. Bell's conviction at the time, that Fermat's unmatched insight
into the character of numbers did not mislead him as to the existence of
a proof, was vindicated recently when a proof was finally found.

Bell's infectious love of number, his sense of the large sweep of
mathematical history, and his generous anger at the fools and poopheads
who all too densely populate that history, are his great legacy to
posterity.

Postscript 2003

The above notice happened to catch the eye of ETB's grandson, Lyle Bell,
the only son of Taine Bell, who commented that ETB ``would have loved
the last paragraph.'' We asked Lyle what ETB was like as a grandfather.
Here, with his permission, is his answer, along with notes on ETB's last
days. WE should add by way of clarification that ``Romps'' was ETB's
name for himself, and ``Toby'' was his name for his wife Jessie.
\begin{quote}
``E.T. Bell, known as Pop Romps to me (he refused to be called Grandpa),
brings back some interesting and humorous memories. You asked how he was
as a grandfather. The answer is that he tolerated me and my two sisters
reasonably well. He wasn't the type to sit his grandkids on his lap. I
believe he was pretty fond of Laurie, the older of the two (both younger
than me). From what I know, he always took more of a liking to females
than males (that reminds me of the fact that he apparently did not want
any children, his wife ``tricked'' him into it, and he was unhappy that
he had a boy rather than a girl). We visited him in Pasadena maybe two
or three times that I can remember. I remember a modest home with walls
full of books and lots of cigar smoke. He had quite a large garden and a
really neat study in a small structure in the middle of the garden. He
visited us in Watsonville a few times when he was healthy. I remember
one time when I felt I really hadn't spent any time with him, and asked
to stay home from school for the day. We sat in the living room
together, no doubt awkwardly for him. I don't remember anything about
the visit except that I was just learning to multiply at the time, and
he asked me to figure out what 7 x 7 is. It took me a while to get 49,
and to this day those are my favorite numbers.

Some time in late 1958 or early 1959 he was still living in Pasadena
when he fell off a chair he was standing on to change a light bulb and
broke his arm. My parents (both physicians) decided to bring him up to
the Watsonville Hospital where they could keep an eye on him while he
recovered. Then they decided he wasn't fit to live by himself and that
he should live in a rest home in Watsonville. When they announced their
plans, he made is abundantly clear that he had no intention of living
with a bunch of old farts. He would stay right where he was, end of
discussion! He was moved to a private room close to the nurses' station.
He flirted with the nurses, smoked cigars, filled his room with books,
worked on The Last Problem, and drank Scotch. He was quite popular with
the nurses. On one occasion he caught his bed on fire as a result of
careless smoking. He was reprimanded by the hospital administration and
told he could not smoke without a nurse being present. He said, ``Great,
I will enjoy more company!'' and continued to smoke when he wanted to.

I was in the sixth grade when he passed away. As I was getting on my
bicycle to ride to school, my Mom came out to tell me that Pop Romps had
died the night before. She told me to go to school. That evening I
remember having a short, awkward conversation with my Dad and telling
him I was sorry that Pop Romps had died. There was NO other discussion
or acknowledgement of his death within the family at the time. There was
no way that my Dad would get involved in with kind of service because
his religious upbringing was totally non-existent. When he was in the
third grade, he asked his parents what the big plus sign on the building
was for!

When I was in college, it occurred to me that I never knew what happened
after Pop Romps died. I asked my Mom, and she told me he was still in
her closet. He had left very specific instructions that he was to be
cremated and his ashes spread at the base of the rock on a hillside in
Yreka, California, where he had proposed to his wife, Toby. He had
spread her ashes there. He left directions and a photo of the location.
Yreka is in the northernmost part of California. My parents did not have
an occasion to go up there and didn't want to make a special trip, so he
stayed in the closet. They made the trip in the early 1970's. They
arrived early on a foggy morning and found the rock near a swing set in
the back yard of a nice home. Dad hopped the fence and did the deed. The
ashes were very light in color and contained bone fragments. They didn't
exactly blend with the green grass, which was covered with dew. He tried
to work them into the grass with his feet and just made matters worse.
He got back in the car and drove out of town fast before somebody
spotted the rather suspicious activity. Pop Romps was no doubt reveling
through the whole ordeal.''

\end{quote}

These pages are copyright 2001-by E. Bruce Brooks

\textbf{Group Projects- Math 528}
\begin{enumerate}
\def\labelenumi{\arabic{enumi}.}

\item
  Paths and Binomial Coefficients
\item
  On Triangular Numbers
\item
  Fibonacci Numbers
\item
  Generalized Pascal Triangles
\item
  A Pascal-like Triangle of Eulerian Numbers
\item
  Leibnitz Harmonic Triangle
\item
  The Twelve Days of Christmas

\end{enumerate}

PROJECT 1: PATHS AND BINOMIAL COEFFICIENTS
\begin{enumerate}
\def\labelenumi{\arabic{enumi}.}

\item
  A path consists of vertical and horizontal line segments of length 1.
  Determine the number of paths
\begin{enumerate}
  \def\labelenumii{\alph{enumii}.}

  \item
    of length 7 from the origin\(\ \left( 0,\ 0 \right)\) and
    \((4,\ 3)\).
  \item
    of length \(a + b\) from \(\left( 0,\ 0 \right)\) and \((a,b)\).

\end{enumerate}
\item
  A lattice point in the plane is a point \((m,\ n)\) with
  \(m,\ n \in Z\).
\begin{enumerate}
  \def\labelenumii{\alph{enumii}.}

  \item
    How many paths of length 4 are there from \((0,\ 0)\) to lattice
    points on the line \(y = - x + 4\)?
  \item
    Determine the number of paths of length 10 from \((0,\ 0)\) to the
    lattice points on \(y = 10 - x\).

\end{enumerate}
\item
  Give a geometrical (path argument) for the identity
  \(2^{n} =
\begin{pmatrix}
  n
  0
  \end{pmatrix}
 +
\begin{pmatrix}
  n
  1
  \end{pmatrix}
 + \ \ldots +
\begin{pmatrix}
  n
  n
  \end{pmatrix}
\ \)using the idea in \#2.
\item
  Generalize \#1 to three dimensions; introduce obstructions; use
  inclusion-exclusion.
\item
  Give a path proof for \(\begin{pmatrix}
  n
  k
  \end{pmatrix}
 =
\begin{pmatrix}
  n - 1
  k
  \end{pmatrix}
 +
\begin{pmatrix}
  n - 1
  k - 1
  \end{pmatrix}
\).
\item
  How many lattice paths from \(\left( 0,\ 0 \right)\) to all lattice
  points on the line \(x + 2y = n\) are there?
\item
  Determine the number of lattice paths from
  \(\left( 0,\ 0,\ 0 \right)\) to lattice points on that portion of the
  plane \(x + y + z = 2\) that lies in the first octant.
\item
  Generalize \#7.
\item
  Determine the number of lattice paths from \(\left( 0,\ 0 \right)\) to
  \((n,\ n)\) that do not cross over (they may touch) the
  line\(\ y = x\).

\end{enumerate}

PROJECT 2: ON TRIANGULAR NUMBERS

Let \(T_{n} = \frac{n\left( n + 1 \right)}{2}\) denote the
\(n^{\text{th}}\) triangular number.
\begin{enumerate}
\def\labelenumi{\arabic{enumi}.}

\item
  Give a geometric interpretation for
  \(T_{n} + T_{n + 1} = \left( n + 1 \right)^{2}.\)
\item
  Verify that \(T_{n} + T_{m} + mn = T_{m + n}\) algebraically.
\item
  Give a geometric interpretation of the formula in \#2.
\item
  Verify that \(T_{n}^{2} + T_{n - 1}^{2} = T_{n^{2}}\).
\item
  Verify the following:
\begin{enumerate}
  \def\labelenumii{\alph{enumii}.}

  \item
    \(3T_{n} + T_{n + 1} = T_{2n + 1}\)
  \item
    \(3T_{n} + T_{n - 1} = T_{2n}\) (also give a geometrical proof)

\end{enumerate}
\item
  Determine the values \(T_{6},\ T_{66},\ T_{666},\ T_{6666},\ \ldots\).
  Make a general statement and prove it.
\item
  Determine the formula for
  \(T_{3} + T_{6} + T_{9} + \ \ldots + T_{3n}.\)
\item
  Verify:
\begin{enumerate}
  \def\labelenumii{\alph{enumii}.}

  \item
    \(1 + 3 + 6 + 10 + 15 = 1^{2} + 3^{2} + 5^{2}\)
  \item
    \(1 + 3 + 6 + 10 + 15 + 21 = 2^{2} + 4^{2} + 6^{2}\)
  \item
    Generalize and prove your generalization.

\end{enumerate}

\end{enumerate}

PROJECT 3: FIBONACCI NUMBERS
\begin{enumerate}
\def\labelenumi{\arabic{enumi}.}

\item
  Briefly address history.
\item
  Using a variety of proof methods (geometric, induction, matrices,
  recursion, telescoping sum) prove identities such as:
\begin{enumerate}
  \def\labelenumii{\alph{enumii}.}

  \item
    \(F_{1} + F_{2} + \ \ldots + F_{n} = F_{n + 2} - 1\)
  \item
    \(F_{1}^{2} + F_{2}^{2} + \ \ldots + F_{n}^{2} = F_{n}F_{n + 1}\)
  \item
    \(F_{n + 1}^{2} - F_{n}F_{n + 2} = \left( - 1 \right)^{n}\)
  \item
    \(F_{n + 2} =
\begin{pmatrix}
    n + 1
    0
    \end{pmatrix}
 +
\begin{pmatrix}
    n
    1
    \end{pmatrix}
 +
\begin{pmatrix}
    n - 1
    2
    \end{pmatrix}
 + \ \ldots\)

\end{enumerate}
\item
  Prove the ``Binet Formula,''
  \(F_{n} = \frac{\alpha^{n} - \beta^{n}}{\alpha - \beta}\), where
  \(\alpha,\ \beta\ \)satisfy \(x^{2} - x - 1 = 0\).
\begin{enumerate}
  \def\labelenumii{\alph{enumii}.}

  \item
    By induction.
  \item
    Using generating series.

\end{enumerate}

\end{enumerate}
FIX
% \[
% begin{pmatrix}
% n \\
% 0 \\
% \end{pmatrix} F_{0} + \begin{pmatrix}
% n \\
% 1 \\
% \end{pmatrix} F_{1} + \ \ldots + \begin{pmatrix}
% n \\
% n \\
% \end{pmatrix} F_{0} = \ F_{2n}
% \]
\begin{enumerate}
\def\labelenumi{\arabic{enumi}.}

\item
  Investigate the geometric interpretations of the \(F_{n}\).

\end{enumerate}

\[
_{n} = \left\lbrack \begin{pmatrix}
n \\
1 \\
\end{pmatrix} + \begin{pmatrix}
n \\
3 \\
\end{pmatrix}5 + \begin{pmatrix}
n \\
5 \\
\end{pmatrix}5^{2}\ldots \right\rbrack + \left\lbrack \begin{pmatrix}
n \\
0 \\
\end{pmatrix} + \begin{pmatrix}
n \\
2 \\
\end{pmatrix} + \begin{pmatrix}
n \\
4 \\
\end{pmatrix} + \ \ldots \right\rbrack.
\]
\begin{enumerate}
\def\labelenumi{\arabic{enumi}.}

\item
  Let \(Q =
\begin{pmatrix}
  1 & 1
  1 & 0
  \end{pmatrix}
\). Prove:
\begin{enumerate}
  \def\labelenumii{\alph{enumii}.}

  \item
    \(Q^{n} =
\begin{pmatrix}
    F_{n + 1} & F_{n}
    F_{n} & F_{n - 1}
    \end{pmatrix}
\).
  \item
    \(Q^{2} = Q + I\); use \(Q^{2n}\ \)to prove the identity in \#4.
  \item
    Use \(Q^{2n + 1} = Q^{n}Q^{n + 1}\) to show
    \(F_{2n + 1} = F_{n + 1}^{2} - F_{n}^{2}\ \)and other identities.

\end{enumerate}
\item
  Show that if \(x^{2} = x + 1\) then \(x^{n} = F_{n}x + F_{n - 1}\)
  for\(\ n \geq 2\). Use this result to prove the Binet Formula, namely
  that \(F_{n} = \frac{\alpha^{n} - \beta^{n}}{\alpha - \beta}\).
\item
  Investigate Lucas, Tribonacci, Tetranacci numbers.

\end{enumerate}

References: The Fibonacci Quarterly; The Golden Section by Garth E.
Runion; The Divine Proportion by H.E. Huntley

PROJECT 4: GENERALIZED PASCAL TRIANGLES
\begin{enumerate}
\def\labelenumi{\arabic{enumi}.}

\item
  Expand\({\ \left( 1 + x + x^{2} \right)}^{2},\ \left( 1 + x + x^{2} \right)^{3},\ \left( 1 + x + x^{2} \right)^{4}\).
\item
  Organize the coefficients as in the following triangle:

\end{enumerate}
\begin{quote}
1

1 1 1

1 2 3 2 1

1 3 6 7 6 3 1

1 4 10 16 19 16 10 4 1

\end{quote}
\begin{enumerate}
\def\labelenumi{\arabic{enumi}.}

\item
  Investigate the properties of this triangle:
\begin{enumerate}
  \def\labelenumii{\alph{enumii}.}

  \item
    Produce the next row.
  \item
    Give a recursion that produces elements in this triangle.
  \item
    What are the row sums? Prove it!
  \item
    What is the sum \(1^{2} + 2^{2} + 3^{2} + 2^{2} + 1^{2}\)?
    Generalize and prove.
  \item
    Is there a hockey stick theorem?

\end{enumerate}
\item
  Investigate 3-dimensional versions of
  \(\left( 1 + x + x^{2} \right)^{n}\) and
  \(\left( x + y + z \right)^{n}.\)
\item
  Investigate \(\left( 1 + x + x^{2} + x^{3} \right)^{n}\).
\item
  Investigate \(\left( 1 + x + x^{2} + \ \ldots + x^{k} \right)^{n}\).
\item
  Where do the ``Fibonacci'' numbers get involved in the Pascal
  triangle?
\item
  Where do the ``Tribonacci'' numbers appear in the triangle in \#2?

PROJECT 5: A PASCAL-LIKE TRIANGLE OF EULERIAN NUMBERS

\end{enumerate}
\begin{enumerate}
\def\labelenumi{\arabic{enumi}.}

\item
  Show algebraically and geometrically that\(\
\begin{pmatrix}
  n
  2
  \end{pmatrix}
 +
\begin{pmatrix}
  n + 1
  2
  \end{pmatrix}
 = n^{2}\).
\item
  Show that \(\begin{pmatrix}
  n
  3
  \end{pmatrix}
 + 4
\begin{pmatrix}
  n + 1
  3
  \end{pmatrix}
 +
\begin{pmatrix}
  n + 2
  3
  \end{pmatrix}
 = n^{3}\).
\item
  Use \#1 and the hockey stick theorem to find a nice formula for

\(1^{2} + 2^{2} + 3^{2} + \ \ldots + n^{2}\). Contrast with the
  version usually seen in calculus.
\item
  Use \#2 to find a nice formula for
  \(1^{3} + 2^{3} + 3^{3} + \ \ldots + n^{3}\).
\item
  Express \(n^{4}\ \)as in problems \#1 and \#2.
\item
  Use \#5 to find a nice formula for
  \(1^{4} + 2^{4} + 3^{4} + \ \ldots + n^{4}\).
\item
  Here is an alternate way of accomplishing \#6:

\end{enumerate}

\[
^{4} + 2^{4} + 3^{4} + \ \ldots + n^{4} = a\begin{pmatrix}
n + 1 \\
5 \\
\end{pmatrix} + b\begin{pmatrix}
n + 2 \\
5 \\
\end{pmatrix} + c\begin{pmatrix}
n + 3 \\
5 \\
\end{pmatrix} + d\begin{pmatrix}
n + 4 \\
5 \\
\end{pmatrix}
\]
\begin{quote}
using n=1, 2, 3, 4. This sometimes referred to as ``Polynomial
Fitting.''

\end{quote}
\begin{enumerate}
\def\labelenumi{\arabic{enumi}.}

\item
  Express\(\ n^{3}\ \)as in problems \#1 and \#2.
\item
  Express\(\ 1^{5} + 2^{5} + 3^{5} + \ \ldots + n^{5}\) as a sum of
  multiples of binomial coefficients.
\item
  In problem \#1, \#2, \#5, and \#7, \(n^{2},n^{3},\ n^{4},\ \)and
  \(n^{5}\) were expressed as sums of binomial coefficients with integer
  coefficients. Arrange these expressions in a Pascal-like triangle,
  concentrating on these integer coefficients. The coefficients are
  called Eulerian numbers.
\item
  Investigate the properties of this triangle. Include a discussion of
  row sums, how entries are found, and a possible recursion. Use
  \(\begin{bmatrix}
  n
  k
  \end{bmatrix}
\) for notation.

PROJECT 6: THE LEIBNITZ HARMONIC TRIANGLE

\end{enumerate}
\begin{enumerate}
\def\labelenumi{\arabic{enumi}.}

\item
  State the rule used to form each entry, and produce the next two rows.
\item
  What are the ``initial conditions'' needed to generate the entries?
\item
  Use \(\begin{bmatrix}
  n
  k
  \end{bmatrix}
\) as the notation for entries and state properties
  analogous to those found in the Pascal Triangle. Include closed
  formulas for each entry, row sums (alternating also), hockey stick
  theorem, hexagon property, sum of squares of row entries, etc.
  Indicate how partial fractions help in verifying the (infinite) hockey
  stick theorem.
\item
  See if you can find any papers on this topic. Check PI MU EPSILON
  JOURNAL, THE MATHEMATICAL GAZETTE, MATHEMATICS MAGAZINE,
  DELTA-UNDERGRADUATE MATHEMATICS JOURNAL, etc.

\end{enumerate}

PROJECT 7: THE TWELVE DAYS OF CHRISTMAS (in the Discrete Math Style)

According to the song, THE TWELVE DAYS OF CHRISTMAS the ``true love''
received the following number of (not so practical) gifts:
\begin{longtable}[]{@{}lll@{}}
\toprule
Day's Gift & Number received & Total\tabularnewline
\midrule
\endhead
First & 1 X 12 & 12\tabularnewline
Second & 2 X 11 & 22\tabularnewline
Third & 3 X 10 & 30\tabularnewline
Fourth & 4 X 9 & 36\tabularnewline
Fifth & 5 X 8 & 40\tabularnewline
Sixth & 6 X 7 & 42\tabularnewline
Seventh & 7 X 6 & 42\tabularnewline
Eighth & 8 X 5 & 40\tabularnewline
Ninth & 9 X4 & 36\tabularnewline
Tenth & 10 X 3 & 30\tabularnewline
Eleventh & 11 X 2 & 22\tabularnewline
Twelfth & 12 X 1 & 12\tabularnewline
& & Grand Total = 364\tabularnewline
\bottomrule

\end{longtable}

1

2 2

3 4 3

4 6 6 4

5 8 9 8 5

The first five rows of a triangle are given above.
\begin{enumerate}
\def\labelenumi{\arabic{enumi}.}

\item
  Make the next few rows.
\item
  Where have you seen this triangle before?
\item
  Experiment with factoring each integer in the 6\textsuperscript{th}
  row. Repeat with other rows.
\item
  What are the entries in row 12?
\item
  Where does the expression
  \(1 \bullet n + 2\left( n - 1 \right) + 3\left( n - 2 \right) + \ \ldots + n \bullet 1\)
  show up in the triangle?
\item
  Explain why
  \(1 \bullet n + 2\left( n - 1 \right) + 3\left( n - 2 \right) + \ \ldots + n \bullet 1 =
\begin{pmatrix}
  n + 2
  3
  \end{pmatrix}
\).

\end{enumerate}

FOUR (OR EIGHT) DISTRIBUTION PROBLEMS

The formulas derived earlier for the number of \emph{onto} functions f
from a domain \(\{ x_{1},\ x_{2},\ldots,\ x_{m}\}\) to a codomain
\(\{ y_{1},\ y_{2},\ldots,\ y_{n}\}\) can be used as a model for the
following distribution problem: in how many ways can you place n
different objects into k different bins, not allowing empty bins? Some
texts use the word distinguishable instead of different. We will also
use the word similar to connote the same meaning as indistinguishable.

The concept of \emph{compositions} of n into k parts likewise serves as
a model for the following distribution problem: in how many ways can you
place n similar objects into k different boxes either allowing empty
boxes or not? As this second class of problem is a little easier let's
analyze it first.

RESULT: The number of ways of placing n similar objects into k different
bins not allowing empty bins is \(\begin{pmatrix}
n - 1
k - 1
\end{pmatrix}
\). Denote the k different bins by
\(x_{1},\ x_{2},\ldots,\ x_{k}\). The number of solutions to
\(x_{1} + x_{2} + \ \ldots + x_{k} = n\) where each \(x_{i} \geq 1\) is
the number of such distributions.

RESULT: The number of ways of placing n similar objects into k different
bins allowing empty bins is \(\begin{pmatrix}
n + k - 1
k - 1
\end{pmatrix}
\). The number of solutions to
\(x_{1} + x_{2} + \ \ldots + x_{k} = n\) where each \(x_{i} \geq 0\) is
the number of such distributions. Notice here that allowing, \(x_{2}\),
for example, to be 0 is the same as saying that no objects will be
placed in bin \#2.

Now let's turn to the first problem. Using the Principle of
Inclusion-Exclusion (PIE) we see that the number of functions from \{1,
2, 3, 4, 5\} \emph{onto} \{a, b, c\} is
\(3^{5} - 3 \bullet 2^{5} + 3 \bullet 1^{5}\). Similarly, the number of
functions from \{1, 2, 3, 4, 5, 6\} \emph{onto} \{a, b, c, d\} is

\(4^{6} - 4 \bullet 3^{6} + 6 \bullet 2^{6} - 4 \bullet 1^{6} =
\begin{pmatrix}
4
4
\end{pmatrix}
4^{6} -
\begin{pmatrix}
4
3
\end{pmatrix}
3^{6} +
\begin{pmatrix}
4
2
\end{pmatrix}
2^{6} -
\begin{pmatrix}
4
1
\end{pmatrix}
1^{6}.\) PIE can be used to show that, in general, the
number of functions from \(\{ x_{1},\ x_{2},\ \ldots,\ x_{n}\}\)
\emph{onto} \(\{ y_{1},\ y_{2},\ldots,\ y_{k}\}\) is

\[
\left( n,k \right) = k^{n} - \begin{pmatrix}
k \\
1 \\
\end{pmatrix}\left( k - 1 \right)^{n} + \begin{pmatrix}
k \\
2 \\
\end{pmatrix}\left( k - 2 \right)^{n} - \begin{pmatrix}
k \\
3 \\
\end{pmatrix}\left( k - 3 \right)^{n} + \ \ldots + \left( - 1 \right)^{k - 1}\begin{pmatrix}
k \\
k - 1 \\
\end{pmatrix}1^{n}
\]

Each onto functions characterizes a distribution problem as follows: the
following onto function
\begin{longtable}[]{@{}llllll@{}}
\toprule
x & 1 & 2 & 3 & 4 & 5\tabularnewline
\midrule
\endhead
f(x) & a & a & b & c & b\tabularnewline
\bottomrule

\end{longtable}

can be considered as a model for distributing the two different objects
1 and 2 into a bin \emph{a} , the two different objects 3 and 5 into bin
\emph{b} and 4 into bin \emph{c} . Since each of the complete inverse
images \(f^{- 1}\left( a \right),\ f^{- 1}\left( b \right),\) and
\(f^{- 1}(c)\) is non-empty (the function being onto guarantees this)
this model counts distributions that do not allow empty bins. This
establishes the following distribution result.

RESULT: The number of ways of placing \emph{n} different objects into
\emph{k} different bins with no empty bins is \(B(n,k)\).

RESULT: The number of ways of placing n different objects into k
different bins in \(k^{n}.\)

Without the restriction of ``no empty bins'' this is just a matter of
saying that there are k places for disposing of each of the n objects.
This is also the number of functions from
\(\{ x_{1},\ x_{2},\ \ldots,\ x_{n}\}\) \emph{to}
\(\{ y_{1},\ y_{2},\ldots,\ y_{k}\}\), ignoring the restriction that
they must be onto.

Finally, we look at what happens if the n objects are different but the
k bins are now similar, still with no empty bins allowed. Since there
are k! ways to label the k bins, there are k! ways to convert similar
bins into different bins. The number of distributions of this type is
therefore \(\frac{1}{k!}B(n,k)\). For convenience of notation, let's
call this last expression: \(S\left( n,k \right).\)

RESULT: The number of ways of placing n different objects into k similar
bins with no empty bins is \(S\left( n,k \right).\)

There is one additional distribution problem that we can analyze using
the machinery developed thus far. In how many ways can you place n
different objects into k similar bins \emph{allowing} empty bins. Using
the sum rule we can look at the following disjoint cases:

No box is empty- \(S(n,\ \ k)\)

Exactly one box is empty-\(S(n,\ k - 1)\)
\begin{quote}
Exactly two bins are empty-\(S(n,\ k - 2)\)

\end{quote}

\(\vdots\)

All but one bin are empty-\(S(n,\ 1)\)

RESULT: The number of ways of distributing n different objects into k or
fewer bins is
\(S\left( n,\ 1 \right) + S\left( n,\ 2 \right) + \ \ldots + S(n,\ k)\).

Finally, the theory of linear or integer \emph{partitions} solves the
final distribution problem: In how many ways can you distribute n
similar objects into k similar bins? Recall that a \emph{partition}
\(\left( a_{1},\ a_{2},\ \ldots,\ a_{k} \right)\ \)of an integer n is an
array of integers \(a_{i}\) such that
\(n = a_{1} + a_{2} + \ \ldots + a_{n}\) and
\(a_{1} \geq a_{2} \geq a_{3} \geq \ \ldots \geq a_{k} > 0\). For
example, the 5 partitions of n=4 are 4, 31, 22, 211, 1111. Here, the
distribution 31 is the same as 13 since the bins are similar.

DISTRIBUTIONS OF OBJECTS INTO BINS
\begin{longtable}[]{@{}lllll@{}}
\toprule
n objects & & & k bins &\tabularnewline
\midrule
\endhead
& & & Allow empty & \(P_{k}(n)\)\tabularnewline
S & (Partitions of Integers) & S & &\tabularnewline
& & & Do not & \(P_{k}(n - k)\)\tabularnewline
& & & Allow empty & \(\begin{pmatrix}
n + k - 1 \\
k - 1 \\
\end{pmatrix}\)\tabularnewline
S & (Compositions) & D & &\tabularnewline
& & & Do not & \(\begin{pmatrix}
n - 1 \\
k - 1 \\
\end{pmatrix}\)\tabularnewline
& & & Allow empty &
\(S\left( n,1 \right) + S\left( n,2 \right) + \ \ldots + S(n,k)\)\tabularnewline
D & (Set Partitions) & S & &\tabularnewline
& & & Do not & \(\frac{1}{k!}B(n,k)\)\tabularnewline
& & & Allow empty & \(k^{n}\)\tabularnewline
D & (Surjections) & D & &\tabularnewline
& & & Do not & \(B(n,k)\)\tabularnewline
\bottomrule

\end{longtable}

CONVENIENT NOTATION

S = Similar

D = Different

\(B\left( n,k \right) = k^{n} -
\begin{pmatrix}
k
1
\end{pmatrix}
\left( k - 1 \right)^{n} +
\begin{pmatrix}
k
2
\end{pmatrix}
\left( k - 2 \right)^{n} + \ \ldots + \left( - 1 \right)^{k - 1}
\begin{pmatrix}
k
k - 1
\end{pmatrix}
1^{n}\)

\[
\left( n,k \right) = \frac{1}{k!}B\left( n,k \right)\
\]

COMBINATORIAL PROOFS

A combinatorial proof is sometimes used to show that two very different
looking expressions are in fact equal. The technique is as follows.
Refer to the two different looking expressions as the left-hand side
(LHS) and the right-hand side RHS. Create a situation or question that
is answered by the LHS; then show that the RHS also answers the
question. The conclusion is that LHS=RHS.

In the following we present a number of theorems, statements,
identities, etc, and give combinatorial proofs of each. For each result
the reader is urged to attempt an alternate proof for comparison
purposes. Such an alternate proof could be algebraic or geometric in
nature; or one could try a collapsing sum or an induction proof.

\emph{THEOREM 1} \(\begin{pmatrix}
2n
2
\end{pmatrix}
 = 2
\begin{pmatrix}
n
2
\end{pmatrix}
 + n^{2}\)

Split the \(2n\) objects into two groups A and B as shown:

\textbf{. . . . . . . . . . }

A B

First, you can choose 2 objects from a set of \(2n\ \)objects
in\(\
\begin{pmatrix}
2n
2
\end{pmatrix}
\ \)ways. Alternatively, you could select two from group A
in\(\
\begin{pmatrix}
n
2
\end{pmatrix}
\) ways or two from group B in\(\
\begin{pmatrix}
n
2
\end{pmatrix}
\) ways or take one from each in \(n \bullet n = n^{2}\)
ways. Now add\(\
\begin{pmatrix}
n
2
\end{pmatrix}
 +
\begin{pmatrix}
n
2
\end{pmatrix}
 + n \bullet n\) and the result follows. The reader should
attempt an algebraic proof using the factorial formula
for\(\
\begin{pmatrix}
n
k
\end{pmatrix}
\).

\emph{THEOREM 2} \(\begin{pmatrix}
m + n
2
\end{pmatrix}
 -
\begin{pmatrix}
m
2
\end{pmatrix}
 -
\begin{pmatrix}
n
2
\end{pmatrix}
 = mn\)

Suppose you have a group of m men and n women and you want to form
men-women dancing pairs. This can clearly be done in mn ways. Or, you
could choose 2 from the total of m + n and delete the men-men pairs
(there are \(\begin{pmatrix}
m
2
\end{pmatrix}
\ \)of these) and delete the women-women pairs
(also\(\begin{pmatrix}
n
2
\end{pmatrix}
\)of these). The result follows. The reader could attempt
an algebraic proof or perhaps a geometric proof making use of figures
consisting of triangular numbers. Also, as a challenge, the reader could
formulate a similar result involving \(\begin{pmatrix}
a + b + c
3
\end{pmatrix}
\) and a corresponding proof.

\emph{THEOREM 3} \(\begin{pmatrix}
n
0
\end{pmatrix}
 +
\begin{pmatrix}
n
1
\end{pmatrix}
 +
\begin{pmatrix}
n
2
\end{pmatrix}
 + \ \ldots +
\begin{pmatrix}
n
n
\end{pmatrix}
 = 2^{n}\)

Here again we first create a question that is answered by either side of
the given identity. Question: How many subsets does
\(\{ a_{1},\ a_{2},\ldots,\ a_{n}\}\) have? An n-set has
\(\begin{pmatrix}
n
2
\end{pmatrix}
\) subsets. The left-hand side counts these subsets by
their size. There are \(\begin{pmatrix}
n
k
\end{pmatrix}
\ \)subsets of size k.

In this situation the reader might \emph{not} want to try this
algebraically.

\emph{THEOREM 4} \(\begin{pmatrix}
n
1
\end{pmatrix}
 + 2
\begin{pmatrix}
n
2
\end{pmatrix}
 + 3
\begin{pmatrix}
n
3
\end{pmatrix}
 + \ \ldots + n
\begin{pmatrix}
n
n
\end{pmatrix}
 = {n2}^{n - 1}\)

Given a set of n people we can select a committee of size k along with a
chair from that committee in \(k
\begin{pmatrix}
n
k
\end{pmatrix}
\ \)ways. We can select a committee (of size 1, or size 2,
or \ldots{}) and its chair in \(\begin{pmatrix}
n
1
\end{pmatrix}
 + 2
\begin{pmatrix}
n
2
\end{pmatrix}
 + 3
\begin{pmatrix}
n
3
\end{pmatrix}
 + \ \ldots + n
\begin{pmatrix}
n
n
\end{pmatrix}
\ \)ways. Alternatively, we can explain the term
\(n2^{n - 1}\ \)as follows: choose one of the n people to chair any of
the \(2^{n - 1}\)subsets of the remaining \(n - 1\) people.

The reader is invited to investigate one or more of the following
approaches: \(n2^{n - 1}\) looks like a derivative, so try
differentiating \(\left( 1 + x \right)^{n}\); a reverse and add approach
also works; or, first prove \(k
\begin{pmatrix}
n
k
\end{pmatrix}
 = n
\begin{pmatrix}
n - 1
k - 1
\end{pmatrix}
\ \)and then use it.

\emph{THEOREM 5} \(\
\begin{pmatrix}
n
k
\end{pmatrix}
 =
\begin{pmatrix}
n - 1
k
\end{pmatrix}
 +
\begin{pmatrix}
n - 1
k - 1
\end{pmatrix}
\)

\(\begin{pmatrix}
n
k
\end{pmatrix}
\ \)is the number of subsets of
\(\{ a_{1},\ a_{2},a_{3},\ \ldots,\ a_{n}\}\) of size k. Now a subset A
of size k either contains the fixed element\(\ a_{i}\) or it does not.
If A contains \(a_{i}\), the remaining \(k - 1\) elements can be
selected in \(\begin{pmatrix}
n - 1
k - 1
\end{pmatrix}
\) ways. If, on the other hand, A does not contain
\(a_{i}\), you can choose the k elements from the depressed set
\(\left\{ a_{1},\ a_{2},\ \ldots,\ a_{i - 1,}a_{i + 1,\ \ldots,\ }a_{n} \right\}\ \)in
\(\begin{pmatrix}
n - 1
k
\end{pmatrix}
\) ways. Since these two cases are mutually exclusive the
theorem follows.

Once again the reader is invited to attempt an algebraic proof.

\emph{THEOREM 6} Let \(d_{n}\) denote the number of derangements of 1,
2, 3, \ldots{}, n with

\(d_{0} = 1,\ d_{1} = 0.\) Then
\(d_{n} = \left( n - 1 \right)(d_{n - 1}{+ d}_{n - 2})\) for
\(n \geq 2\).

In forming a derangement of 1, 2, 3, \ldots{}, n the integer n can be
placed in any of the \(n - 1\) spots 1, 2, 3, \ldots{}, \(n - 1\), say
spot i. If i goes into spot n there are \(d_{n - 2}\) ways to finish it.
If i does not go into spot n there are \(d_{n - 1}\) ways to complete
the derangement.

The reader can use the principle of inclusion-exclusion to derive a
formula for \(d_{n}\)

from which the new recursion
\(d_{n} = nd_{n - 1} + \left( - 1 \right)^{n}\), and the above
recursion, can be derived.

\emph{THEOREM 7} \(\begin{pmatrix}
n
0
\end{pmatrix}
^{2} +
\begin{pmatrix}
n
1
\end{pmatrix}
^{2} +
\begin{pmatrix}
n
2
\end{pmatrix}
^{2} + \ \ldots +
\begin{pmatrix}
n
n
\end{pmatrix}
^{2} =
\begin{pmatrix}
2n
n
\end{pmatrix}
\)

Given a group of \(2n\) people consisting of n men and n women, in how
many ways can one choose a group of n people? The answer to that
question is just\(\begin{pmatrix}
2n
n
\end{pmatrix}
\), the right side of the identity in question. One could
also form the group of n people in the following way: choose 0 men and n
women in

\(\begin{pmatrix}
n
0
\end{pmatrix}
\begin{pmatrix}
n
n
\end{pmatrix}
 =
\begin{pmatrix}
n
0
\end{pmatrix}
^{2}\ \)ways, or, choose 1 man and \(n - 1\) women in

\(\begin{pmatrix}
n
1
\end{pmatrix}
\begin{pmatrix}
n
n - 1
\end{pmatrix}
 =
\begin{pmatrix}
n
1
\end{pmatrix}
^{2}\)ways, or choose 2 men and \(n - 2\) women in

\(\begin{pmatrix}
n
2
\end{pmatrix}
\begin{pmatrix}
n
n - 2
\end{pmatrix}
 =
\begin{pmatrix}
n
2
\end{pmatrix}
^{2}\) ways and so on. Now add these disjoint cases.

An alternate algebraic proof is less interesting: Extract the
coefficient of \(x^{n}\) from both sides
of.\(\left\lbrack \left( x + 1 \right)^{n} \right\rbrack^{2} = \left( x + 1 \right)^{2n}\)

\emph{THEOREM 8} \(\begin{pmatrix}
2
2
\end{pmatrix}
 +
\begin{pmatrix}
3
2
\end{pmatrix}
 +
\begin{pmatrix}
4
2
\end{pmatrix}
 + \ \ldots +
\begin{pmatrix}
n
2
\end{pmatrix}
 =
\begin{pmatrix}
n + 1
3
\end{pmatrix}
\)

The term \(\begin{pmatrix}
n + 1
3
\end{pmatrix}
\) is the number of binary strings of length \(n + 1\)
consisting of three 1's (and the rest 0's). The left hand side counts
these by where in the string the left- most 1 appears. Let
\(a_{1}\ a_{2}a_{3}\ldots a_{n + 1}\) be a string of length \(n + 1\).
There are \(\begin{pmatrix}
n
2
\end{pmatrix}
\) strings when \(a_{1} = 1,\
\begin{pmatrix}
n - 1
2
\end{pmatrix}
\) strings when \(a_{2} = 1\) is the leftmost 1, \ldots{},
and \(\begin{pmatrix}
2
2
\end{pmatrix}
\)strings when \(a_{n - 1} = 1\ \)is the leftmost 1. In
this last case the string looks like 000 \ldots{} 0111.

Attempting a proof by mathematical induction is an easy option. An
algebraic approach is not!

\emph{THEOREM 9} The number of positive integers that have their digits
in strictly increasing order is \(2^{9} - 1\). Include single digit
numbers.

There are \(\begin{pmatrix}
9
1
\end{pmatrix}
\) single digit type, \(\begin{pmatrix}
9
2
\end{pmatrix}
\) double digit type (just select 2 of the 9 digits 1, 2,
3, \ldots{}, 9 and arrange in order), \ldots{}, and so on to see that
there are \(\begin{pmatrix}
9
9
\end{pmatrix}
\) nine digit type. The total is \(\begin{pmatrix}
9
1
\end{pmatrix}
 +
\begin{pmatrix}
9
2
\end{pmatrix}
 +
\begin{pmatrix}
9
3
\end{pmatrix}
 + \ \ldots +
\begin{pmatrix}
9
9
\end{pmatrix}
 = 2^{9} - 1.\)

Here is an alternative, more clever, proof. Look at 123456789. Any
increasing number can be made by deleting \emph{any} subset of digits,
except all of them. There are \(2^{9} - 1\)such subsets. For example,
delete the subset \{2, 4, 7, 9\} and you get 13568. Combining these two
approaches actually gives you a nice proof that \(\begin{pmatrix}
9
0
\end{pmatrix}
 +
\begin{pmatrix}
9
1
\end{pmatrix}
 +
\begin{pmatrix}
9
2
\end{pmatrix}
 + \ \ldots +
\begin{pmatrix}
9
9
\end{pmatrix}
 = 2^{9}\).

\emph{THEOREM 10} \(\begin{pmatrix}
3n
3
\end{pmatrix}
 = 3
\begin{pmatrix}
n
3
\end{pmatrix}
 + 6n
\begin{pmatrix}
n
2
\end{pmatrix}
 + n^{3}\)

This one is a little tougher. First rewrite as \(n^{3} =
\begin{pmatrix}
3n
3
\end{pmatrix}
 - 3
\begin{pmatrix}
n
3
\end{pmatrix}
 - 6n
\begin{pmatrix}
n
2
\end{pmatrix}
\). Suppose you have n men, n women and n children and you
want to select triples consisting of one man, one woman and one child.
There are \(n^{3}\) ways to do this, just pick one from each group.
Alternatively, select 3 of the 3n people in \(\begin{pmatrix}
3n
3
\end{pmatrix}
\) ways and delete the ``bad'' ones. Delete the ones where
you selected all three from one group -- there are \(3
\begin{pmatrix}
n
3
\end{pmatrix}
\)of these. Now delete those where you had two from one
group and one from another -- there are \(2n
\begin{pmatrix}
n
2
\end{pmatrix}
 + 2n
\begin{pmatrix}
n
2
\end{pmatrix}
 + 2n
\begin{pmatrix}
n
2
\end{pmatrix}
\) of these.

\emph{THEOREM 11}
\(1 \bullet 1! + 2 \bullet 2! + 3 \bullet 3! + \ \ldots + n \bullet n! = \left( n + 1 \right)! - 1\)

In how many ways can you arrange the n+1 numbers 0, 1, 2, \ldots{}, n so
that they are \emph{not} in ascending order? The answer is
\(\left( n + 1 \right)! - 1\) since 0, 1, 2, \ldots{}, n is the
\emph{only} arrangement in ascending order. Now lets separate into
cases. Let \(a_{0},\ a_{1},\ a_{2},\ \ldots,\ a_{n}\ \)represent an
arrangement of these n+1 numbers. If \(a_{0} \neq 0\), there are n
choices left for \(a_{0}\), and then n! ways to fill out
\(a_{1},\ a_{2},\ \ldots,\ a_{n}\) for a total of \(n \bullet n!\). Now
let \(a_{0} = 0\) but \(a_{1} \neq 1\). There are \(n - 1\) choices for
\(a_{1}\) and \(\left( n - 1 \right)!\ \)ways to complete for a total of
\(\left( n - 1 \right)\left( n - 1 \right)!\). Now continue with
\(a_{0} = 1,\ a_{1} = 1\) but \(a_{2} \neq 2\). There are
\(\left( n - 2 \right)\left( n - 2 \right)!\) ways, and so on.

The reader should attempt a collapsing sum or induction proof.

\emph{THEOREM 12}
\(1 \bullet n + 2\left( n - 1 \right) + 3\left( n - 2 \right) + \ \ldots + n \bullet 1 =
\begin{pmatrix}
n + 2
3
\end{pmatrix}
\)

Let \(S = \{ 0,\ 1,\ 2,\ \ldots,\ n,\ n + 1\}\). The number of subsets
of S of size 3 is \(\begin{pmatrix}
n + 2
3
\end{pmatrix}
\). Each one looks like \{a, b, c\} with a \textless{} b
\textless{} c, Let's count these by looking at the size of the middle
element b. If b=1 , there is one choice for a, namely a=0 and n choices
for c for a total of \(1 \bullet n\). If b=2 there are 2 choices for a,
and \(n - 1\) choices for c for a total of \(2(n - 1)\). If b=3 the
total is

\(3(n - 2)\), and so on. The total derived by looking at cases is

\(1 \bullet n + 2\left( n - 1 \right) + 3\left( n - 2 \right) + \ \ldots + n \bullet 1\)
and this must equal \(\begin{pmatrix}
n + 2
3
\end{pmatrix}
\) since the cases are disjoint.

\emph{\\
}

\emph{THEOREM 13} \(k
\begin{pmatrix}
n
k
\end{pmatrix}
 = n
\begin{pmatrix}
n - 1
k - 1
\end{pmatrix}
\)

Suppose you have a group of n people and you wish to form a subcommittee
of k people with one of those k people to serve as chair. Choose the
subcommittee in \(\begin{pmatrix}
n
k
\end{pmatrix}
\) ways and the chair in k ways. The product rule
gives\(\text{\ k}
\begin{pmatrix}
n
k
\end{pmatrix}
\) as the number of ways of selecting such a chaired
subcommittee.

Alternatively, you could first choose any one of the n people to serve
as chair and then fill out the committee in \(\begin{pmatrix}
n - 1
k - 1
\end{pmatrix}
\) ways. There are \(n
\begin{pmatrix}
n - 1
k - 1
\end{pmatrix}
\) ways to select a chaired subcommittee. Hence
\(k
\begin{pmatrix}
n
k
\end{pmatrix}
 = n
\begin{pmatrix}
n - 1
k - 1
\end{pmatrix}
\)

The reader should attempt the easier algebraic technique by converting
\(\begin{pmatrix}
n
k
\end{pmatrix}
\) to factorial form.

\emph{THEOREM 14} \(P\left( n,k \right) = k!
\begin{pmatrix}
n
k
\end{pmatrix}
\)

Questions -- How many permutations are there of k objects chosen from a
collection of n objects? The LHS answers the question. There are
\(P\left( n,k \right) = n\left( n - 1 \right)\left( n - 2 \right)...(n - k + 1)\)
ways. Alternatively, one could first choose the k objects from the n
objects in \(\begin{pmatrix}
n
k
\end{pmatrix}
\) ways and then permute these in k! ways.

\emph{\\
THEOREM 15} \(n2^{n - 1} = 1
\begin{pmatrix}
n
1
\end{pmatrix}
 + 2
\begin{pmatrix}
n
2
\end{pmatrix}
 + 3
\begin{pmatrix}
n
3
\end{pmatrix}
 + \ \ldots + n
\begin{pmatrix}
n
n
\end{pmatrix}
\)

Contrast this discussion with that presented in THEOREM 4. Look at the
set of the first \(2^{n}\) nonnegative
integers\(\ 0,\ 1,\ 2,\ \ldots,\ 2^{n} - 1\). When you convert each to
binary form what is the total number of 1s written? This binary list
will look like the standard listing in \(B^{n}\) the set of all binary
strings of length n. For n=3
\(B^{3} = \{ 000,\ 001,\ 010,\ 011,\ 100,\ 101,\ 110,\ 111\}\). In
\(B^{n}\) each string has length n and there are \(2^{n}\) of them. But
half of the \(n \bullet 2^{n}\) symbols are 1's. Then the total number
of 1's is \(n2^{n - 1}\). Alternatively, we could consider each string
and count those with one 1, then those with two 1's, etc. There are
\(1 \bullet
\begin{pmatrix}
n
1
\end{pmatrix}
\) with one 1, \(2
\begin{pmatrix}
n
2
\end{pmatrix}
\) total 1's in those binary numbers with exactly two 1's,
\(3
\begin{pmatrix}
n
3
\end{pmatrix}
\) in those with exactly three 1's , and so on. The total
is \(1
\begin{pmatrix}
n
1
\end{pmatrix}
 + 2
\begin{pmatrix}
n
2
\end{pmatrix}
 + 3
\begin{pmatrix}
n
3
\end{pmatrix}
 + \ \ldots + n
\begin{pmatrix}
n
n
\end{pmatrix}
\text{.\ }\)The result now follows by equating
\(n2^{n - 1}\) to this sum.

\emph{THEOREM 16} \(\begin{pmatrix}
n
0
\end{pmatrix}
^{2} +
\begin{pmatrix}
n
1
\end{pmatrix}
^{2} +
\begin{pmatrix}
n
2
\end{pmatrix}
^{2} + \ \ldots +
\begin{pmatrix}
n
n
\end{pmatrix}
^{2} =
\begin{pmatrix}
2n
n
\end{pmatrix}
\)

Let's revisit this identity using equivalence relations. A binary
relation R on the set of all binary strings of length n is defined by
specifying that \(\left( \alpha,\ \beta \right) \in R\) whenever weight
\(\alpha =\) weight \(\beta\). This R is an equivalence relation. For
n=3 there are four different equivalence classes, each containing
strings of weight 0, 1, 2, or 3. The relation R contains
\(1^{2} + 3^{2} + 3^{2} + 1^{2}\) ordered pairs; for example, with
weight 1 each of the three strings 001, 010, 100 can be paired with any
one of those same strings for a total of \(3^{2} = 9\). These ordered
pairs can be counted in another way. Each ordered pair looks like
\(\left( - - - ,\  - - - \right).\) Place 1's in any three positions,
and 0's in the others. If you take the complement of the entries in the
second coordinate an element of R is produced. Here is what one sequence
of this process looks like:

\(\left( - - - , - - - \right) \rightarrow \left( - 11,\  - - 1 \right) \rightarrow \left( 011,\ 001 \right) \rightarrow (011,\ 110)\).
The reader can check that this process always produces an element of R
and that the case of n=3 extends easily to general n. Conclusion: choose
the n positions for 1's in \(\begin{pmatrix}
2n
n
\end{pmatrix}
\) ways. The result follows.

\emph{THEOREM 17} \(\begin{pmatrix}
n
0
\end{pmatrix}
d_{0} +
\begin{pmatrix}
n
1
\end{pmatrix}
d_{1} +
\begin{pmatrix}
n
2
\end{pmatrix}
d_{2} + \ \ldots +
\begin{pmatrix}
n
n
\end{pmatrix}
d_{n} = n!\) where \(d_{n}\) denotes the
\(\ n^{\text{th}}\ \)derangement number, \(d_{0} = 1,\ d_{1} = 0.\)

The right-hand side, n!, gives the number of permutations of n objects.
So the left-hand side must provide the same enumeration. The left side
partitions the permutations according to how many elements are deranged
(and the rest fixed). The term \(\begin{pmatrix}
n
i
\end{pmatrix}
d_{i} =
\begin{pmatrix}
n
n - i
\end{pmatrix}
d_{\text{i\ }}\)gives the number of permutations of n where
\(n - i\) elements are fixed and the remaining i elements are deranged.
Summing over all i yields

\(\begin{pmatrix}
n
n
\end{pmatrix}
d_{0} +
\begin{pmatrix}
n
n - 1
\end{pmatrix}
d_{1} + \ \ldots +
\begin{pmatrix}
n
0
\end{pmatrix}
d_{n} =
\begin{pmatrix}
n
0
\end{pmatrix}
d_{0} +
\begin{pmatrix}
n
1
\end{pmatrix}
d_{1} + \ \ldots +
\begin{pmatrix}
n
n
\end{pmatrix}
d_{n} = n!\)

\emph{THEOREM 18} \(F_{n + 1} =
\begin{pmatrix}
n
0
\end{pmatrix}
 +
\begin{pmatrix}
n - 1
1
\end{pmatrix}
 +
\begin{pmatrix}
n - 2
2
\end{pmatrix}
 + \ \ldots\) where \(F_{n}\) denotes the
\(\ n^{\text{th}}\ \)Fibonacci number.

Here is a question that might resolve the issue. How many different
brick paths of length n (and width 1) can you make using 1 x 1 bricks
and 1 x 2 bricks? Let a(n) denote the number of such paths of length n.
A few drawings will show that a(1)=1, a(2)=2, a(3)=3, a(4)=5. Since you
can place a 1 x 1 brick in front of all paths of length \(n - 1\) or a 1
x 2 brick in front of all paths of length \(n - 2\) we have that
\(a\left( n \right) = a\left( n - 1 \right) + a\left( n - 2 \right)\).
This recursion, along with the initial conditions, show that
\(a\left( n \right) = F_{n + 1}\), the left-hand side of the identity.

Now lets look at all paths of length n and count them by the number of 1
x 2 bricks. If there are i 1 x 2 bricks there are \(n - i\) total bricks
making up the path of length n. Choose the positions of the i 1 x 2
bricks in \(\begin{pmatrix}
n - i
i
\end{pmatrix}
\) ways. Now sum as i ranges through the values

0, 1, 2, \ldots{}, and obtain \(\begin{pmatrix}
n
0
\end{pmatrix}
 +
\begin{pmatrix}
n - 1
1
\end{pmatrix}
 +
\begin{pmatrix}
n - 2
2
\end{pmatrix}
 + \ \ldots = F_{n + 1}\).

The reader should draw all paths of length \(n = 5\), for example, and
examine the cases with i=0, 1, 2 . The reader could also explore other
proofs.

\emph{THEOREM 19} \(\begin{pmatrix}
m + n
2
\end{pmatrix}
 -
\begin{pmatrix}
m
2
\end{pmatrix}
 -
\begin{pmatrix}
n
2
\end{pmatrix}
 = mn\) (Revisited)

1.

m

n
\begin{figure}[h!]\begin{center}
\includegraphics[width=0.70\columnwidth]{figures/image16/default-figure}
\caption{{Couldn't find a caption, edit here to supply one.%
}}
\end{center}
\end{figure}

2.
There are mn one by one squares in the subdivided

m by n rectangle. Each choice of arrows

(one horizontal, one vertical) specifies one of these squares.

Pick two arrows but don't take two from the top or two from

the side. Then \(\begin{pmatrix}
m + n
2
\end{pmatrix}
 -
\begin{pmatrix}
m
2
\end{pmatrix}
 -
\begin{pmatrix}
n
2
\end{pmatrix}
 = mn\)

3. Take m people in one group and n in another. How many handshakes can
be accomplished? Among the m people there are \(\begin{pmatrix}
m
2
\end{pmatrix}
\) handshakes; among the n people there are
\(\begin{pmatrix}
n
2
\end{pmatrix}
\ \)handshakes and between the two groups, mn. But,
\(\begin{pmatrix}
m + n
2
\end{pmatrix}
\) also represents the total number of handshakes among the
people. Then we get: \(\begin{pmatrix}
m
2
\end{pmatrix}
 +
\begin{pmatrix}
n
2
\end{pmatrix}
 + mn =
\begin{pmatrix}
m + n
2
\end{pmatrix}
\).
\begin{figure}[h!]\begin{center}
\includegraphics[width=0.70\columnwidth]{figures/image18/image18}
\caption{{Couldn't find a caption, edit here to supply one.%
}}
\end{center}
\end{figure}


\end{document}
